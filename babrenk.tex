\documentclass[german,version-2020-11]{uzl-thesis}


% Copy this file as a template for your thesis. You will have to take
% action at all places marked by
%
% !!!!!!!!!!!!!!!!!!!!!!!!!!!!!!!!!!
% !!! Your action is needed here !!!
% !!!!!!!!!!!!!!!!!!!!!!!!!!!!!!!!!!
%
% The first place your action is needed is the first line of this
% document:
%
%
% Language of the thesis:
%
% You must use either 'german' or 'english' above, depending on the
% language used in the main text. This will automatically setup a lot
% of things in the background.
%
%
% Version of the class:
%
% You must specify which version of the thesis class is to be
% used. This is important in case the class style changes in later
% years, but we still want an older thesis to look the same, even when
% things are changed in the class.
%
% Do not change or remove the version-xxxx key.
%
%
% Text encoding:
%
% Your thesis *must* be encoded in utf8 (unicode), which is the
% default in most editors these days. Do *not* change this to latin8.



%%%
%
% Main setup:
%
%%%
%
% You must use the \UzLThesisSetup command to specify numerous things
% about your thesis. This includes the entries on the title page, the 
% abstracts, and the bibliography style. You do so by specifying
% so-called "values" for so-called "keys". For instance, 
% for the key "Autor" you must provide your name as the value. You do
% so by writing 'Autor = {Max Mustermann}', that is, the value is put
% into curly braces. You can use the \UzLThesisSetup command
% repeatedly and the order in which you provide the keys is not
% important. 
%
% Everything shown on the title page must be in German -- even
% if the thesis is written in English! Just insert German text for
% German keys and English text for English keys (like 'Abstract' needs
% English text, while 'Zusammenfassung' needs German text).

\UzLThesisSetup{
  %
  % !!!!!!!!!!!!!!!!!!!!!!!!!!!!!!!!!!
  % !!! Your action is needed here !!!
  % !!!!!!!!!!!!!!!!!!!!!!!!!!!!!!!!!!
  %
  % First, specify the institut or clinic at which the thesis was
  % written. You get the logo file from them (make sure it has the
  % correct size, namely the same as the example). If they do not have
  % a logo, the university's default logo is used.
  %
  % The 'verfasst' gets two arguments. Change the first to {an der}
  % for clinics, as in 'Verfasst = {an der}{Medizinischen Klinik I}'
  %
  Logo-Dateiname        = {uzl-thesis-logo-itcs.pdf},
  Verfasst              = {am}{Institut für Informationssysteme},
  %
  % The titles:
  %
  Titel auf Deutsch     = {
    Topic Modellierung zur Zuordnung von Kundenanfragen an Abteilungen
  }, 
  Titel auf Englisch    = {
    Topic Modeling ... 
  },
  %
  % Author and supervisor:
  % 
  % Note that the 'Betreuer' or 'Betreuerin' is the supervisor, that
  % is, the professor who officially supervises the thesis. If there
  % is also an assistent of the professor who helped (typically a
  % lot), use 'Mit Unterstützung von' to thank that person. If the
  % thesis was mainly written 'externally' at some company or another
  % institute, point this out using 'Weitere Unterstützung'. 
  % 
  % For your own name, do *not* add things like "BSc" or "BSc
  % cand.". For the supervisor, you should normally include
  % "Prof. Dr." or "PD Dr." (ask your supervisor, what is
  % appropriate), but nothing more (so no
  % "Univ.-Prof. Dr. Dr. h.c. mult." unless your supervisor insists).  
  %
  Autor                 = {Leonard Brenk},
  Betreuerin            = {Prof. Dr. Ralf Möller},
  % 
  % Optional: Supporting persons and institutions. The text should be
  % in German, even for an English thesis.
  %
  Mit Unterstützung von = {Dr. Jinghua Groppe, Felix Kuhr, Magnus Bender},
  % 
  %   Weitere Unterstützung = {
  %     Die Arbeit ist im Rahmen einer Tätigkeit bei der Firma Muster GmbH
  %     entstanden.
  %   },
  %
  %
  % Your Degree Programm (Studiengang)
  %
  % Specify 'Bachelorarbeit' or 'Masterarbeit' and the degree
  % programme. Make sure the name of programme is correct and not
  % some abbreviation or some incorrect variant. For instance:
  % 'Medizinische Ingenierwissenschaft', but not 'MIW';
  % 'Medizinische Informatik', but not 'Medizin-Informatik';
  % 'Informatik', but not 'Informatik (SSE)'.
  %
  % Use German names for German programmes and English names for
  % English ones, so 'Infection Biology', not 'Infektionsbiologie'. 
  % For programmes that have a German bachelor and an English master,
  % use the German name for a bachelor thesis and the English name for
  % the master thesis.
  %
  Bachelorarbeit,
  Studiengang           = {Informatik},
  %
  % Date on which the thesis is turned in German, formatted the
  % traditional German way:
  %
  Datum                 = {1. Oktober 2021},
  %
  % The English abstract. You must always provide abstracts in German
  % and in English. 
  %
  Abstract              = {
    It is not easy to write a thesis that does not only advance
    science, but that is also a pleasure to read. While the scientific
    contribution of a thesis is undoubtedly of greater importance, the
    impact of \emph{writing well} should not be underestimated: If
    the person who grades a thesis finds no pleasure in the reading,
    that person are also unlikely to find pleasure in giving outstanding
    grades. A well-written text uses good German or English phrasing with a clear and correct 
    sentence structure and language rhythm, there are no spelling
    mistakes and the author's arguments are presented in a
    clear, logical and understandable manner using well-chosen
    examples and explanations. In addition, a nice-to-read font and a
    pleasing layout are also helpful. The \LaTeX\ class presented in
    this document helps with the latter: It contains a number of
    ready-to-use designs and 
    takes care of many small typographical chores.
  },
  Zusammenfassung       = {
    Es ist nicht leicht, eine Abschlussarbeit so zu schreiben, dass sie
    nicht nur inhaltlich gut ist, sondern es auch eine Freude ist, sie
    zu lesen. Diese Freude ist aber wichtig: Wenn die Person, die die 
    Arbeit benoten soll, wenig Gefallen am Lesen der Arbeit findet,
    so wird sie auch wenig Gefallen an einer guten Note
    finden. Glücklicherweise gibt es einige Kniffe, gut lesbare
    Arbeiten zu schreiben. Am wichtigsten ist zweifelsohne, dass
    die Arbeit in gutem Deutsch oder Englisch verfasst wurde mit klarem
    Satzbau und gutem Sprachrhythmus, dass keine Rechtschreib- oder
    Grammatikfehlern im Text auftauchen und dass die Argumente der
    Autorin oder des Autors klar, logisch, verständlich und gut
    veranschaulicht dargestellt werden. Daneben sind aber auch gut
    lesbare Schriftbilder und ein angenehmes Layout hilfreich. Die Nutzung
    dieser \LaTeX-Vorlage hilft der Schreiberin oder dem Schreiber
    dabei zumindest bei Letzterem: Sie umfasst gute, sofort nutzbare
    Designs und sie kümmert sich um viele typographische
    Details.  
  },
  %
  % Optional: 'Danksagungen' (German) or 'Acknowledgements'
  % (English). Both keys are optional and both have the same effect of
  % adding an acknowledgements text after the abstracts and before the
  % table of contents.
  %
  Acknowledgements      = {
    This is the place where you can thank people and institutions, do
    not try to do this on the title page. The only exception is in
    case you wrote your thesis while working or staying at a company or abroad. Then you
    should use the \Latex{Weitere Unterstützung} key to provide a text
    (in German) that acknowledges the company or foreign
    institute. For instance, you could use texts like »Die Arbeit
      ist im Rahmen einer Tätigkeit bei der Firma Muster GmbH
      entstanden« or »Die Arbeit ist im Rahmen eines
      Forschungsaufenthalts beim Institut für Dieses und Jenes an der
      Universität Entenhausen entstanden«. Do not name and thank
      individual persons from the company or foreign institute on the
      title page, do that here. 
  },
  % Bibliography style: Choose between
  % 
  % 'Alphabetische Bibliographie'
  % for all degree programmes in the natural sciences 
  % 
  % 'Numerische Bibliographie'
  % alternative for all other degree programmes
  % 
  % Either will load biblatex and setup the citation methods and the
  % bibliography styles correctly. You should not mess with them.
  % 
  Alphabetische Bibliographie,
  % Alternatively:
  % Numerische Bibliographie
}




%%%%%%%%%%%%%%%%%%%%
%
% Styling the thesis
%
%%%%%%%%%%%%%%%%%%%%
%
% Creating a visually pleasing layout and choosing fonts is not
% easy. Furthermore, different people have different preferences. Of
% course, for the University of Lübeck, the dean of studies could just
% force everyone to use one specific layout and font, but that seems a
% bit drastic and, also, it seems nice that thesis by different people
% have an individual style even though they all stick to the same
% overall structure.
%
% For these reasons, I (Till Tantau) have spend quite some time on
% designing a flexible layout and styling mechanism for theses.
%
% Basically, the overall structure of the thesis is fixed by the
% thesis class and so are many structural elements. For instance, you
% cannot change the order in which the abstract and table of contents
% are shown, you cannot move the bibliography elsewhere, indeed, the
% bibliography style is also fixed. Likewise, the text on the title
% page is fixed.
%
% Although many things are fixed, you *can* change several other
% things. For instance, you can change the font used for the main
% text, you can change which font is used for titles and headings or
% you can change whether titles and headlines are centered or flushed
% left.
%
% There are many LaTeX packages for changing such things. You are
% kindly asked *not to use them*. Rather, use (only) the options
% offered by the thesis class. All possible choices and combinations
% there have been tested by me and produce nice results; what happens
% with other packages no one knows and might no longer conform to what
% is expected by the university. As you will see, you still have a
% lot of options.
%
%
% Technical note: All styling is done via the command
%
% \UzLStyle{...}
%
% where ... is a key-value list just as for \UzLThesisSetup. The
% difference is just that everything having to do with styling as
% controlled by \UzLStyle, while the more “formal” setup keys are
% controlled by \UzLThesisSetup.
%
%%%
%
% Designs
%
%
% A \emph{design} is a whole set of font and layout options bundled
% together. They have been chosen in such a way that a visually
% pleasing “overall appearance” results.
%
%
% \UzLStyle{computer modern oldschool design}
%
% The look of this design mimics the “classical” way a paper or report
% created with \LaTeX\ looks like: The Computer Modern font is used,
% bold face fonts are used for headlines, only black and white are
% used as colors. This design reminds me of older scientific
% documents, especially from the computer science community where
% \LaTeX\ was used very early.
%
%
% \UzLStyle{computer modern basic design}
%
% A slightly less “oldschool” version of the previous design. It is
% still a classic design in the sense that it uses the Computer Modern
% font and that it still has this “good old \LaTeX” look, but some
% more modern aspects (like colors!) have been added.
%
% Note that this design uses Myriad for the title page (one of the
% “modern aspect”), which means that his font must be installed.
%
%
% \UzLStyle{computer modern scholary design}
%
% In my opinion, this is the ultimate “scholary design”: The thesis
% will look like it has been typeset by hand some 150 years ago and
% then printed by a university press. There is really nothing “modern”
% about it and the word in the name of the design is just part of the
% name of the “Computer Modern” font.
%
%
% \UzLStyle{pagella basic design}
%
% A, well, basic design that uses the Pagella font rather than the
% Computer Modern font. Especially the bold face version of this font
% looks nicer than the Computer Modern counterpart. Also, Pagella,
% while still having a “bookish” look, still feels a bit fresher than
% Computer Modern. 
%
%
% \UzLStyle{pagella centered design}
%
% A variant of the basic Pagella design that centers all
% headlines. A nice alternative to the basic version.
%
%
% \UzLStyle{pagella contrast design}
%
% This design tries to create some visual friction by contrasting the
% sans serif headline font (in bold!) with the main text. I find it a
% visually very interesting combination.
%
%
% \UzLStyle{alegrya basic design}
%
% The third variant of the basic design, this time using the Alegrya
% font. 
%
%
% \UzLStyle{alegrya scholary design}
%
% The Alegrya version of the previous “scholary” design. Unlike the
% Computer Modern version, this design does not look old, but more
% fresh -- while still creating the impression that the text must be
% about a very scientific subject. 
%
%
% \UzLStyle{alegrya stylish design}
%
% The design is quite similar to the scholary version for the Alegrya
% font, but with even more modern additions. “Stylish” is the word
% that comes to my mind.
%
%
\UzLStyle{alegrya modern design}
%
% A design that uses the sans serif version of the Alegrya font for
% the headlines. This is a nice modern overall design.
%
%%%




%%%%%%%%
%
% Now, include the package you need here using \usepackage. 
%
% However, many standard packages are already loaded by the class:
%
% amsmath, amssymb, amsthm, babel, biblatex, csquotes, etoolbox,
% filecontents, fontspec, geometry, hyperref, tikz (with libraries
% arrows.meta, positioning and shapes), varioref, url 
%
% Indeed, in many cases you will not need any extra packages.
%
%%%%%%%


\usepackage{graphicx}
\usepackage{listings}
\usepackage{xcolor}
\usepackage{comment}
\usepackage{tikz}
\usepackage{amsmath}
\usetikzlibrary{arrows, automata, positioning}
\usepackage{pgfplots}
\usepackage{algorithm2e}
\usetikzlibrary{shapes,arrows}
\usetikzlibrary{shapes.geometric, arrows}
\usepackage{algorithm}
\usepackage{algorithmic}
\usepackage{caption}
\usepackage{float}


\begin{document}

%
% The title page and table of contents will be inserted automatically
% here. 
%


\chapter{Einleitung}%%%%%%%%%%%%%%%%%%%%%%%%%%%%%%%%%%%%%%%%%%%%%%%%%%%%%EINLEITUNG
% In a German thesis write: \chapter{Einleitung}


% !!!!!!!!!!!!!!!!!!!!!!!!!!!!!!!!!!
% !!! Your action is needed here !!!
% !!!!!!!!!!!!!!!!!!!!!!!!!!!!!!!!!!
%
% Replace with your own introduction:

% DELETE the following line for your own thesis - it causes trouble!
\lstMakeShortInline[style=code,style=inline,language={[LaTeX]tex},moretexcs={chapter}]|



\section{Contributions of this Thesis}



\section{Related Work}


\section{Sturktur dieser Arbeit}
Der Bereich Data Science und Datenverarbeitung durchläuft aktuell eine starke Welle an Innovation und Veränderung. Deshalb wird in dieser Arbeit zuerst eine Einleitung die Motivation und das Prinzip des Topic Modellings beschreiben. Gefolgt wird die allgemeine Einleitung von der Motivation und Aufgabenstellung in Bezug auf das Anwendungsbeispiel des Zweckverbands Ostholstein (ZVO). Die Anwendung der Algorithmen wird im Laufe der Arbeit mit theoretischen Grundlagen untermauert, die eine klare Notation fordern: diese ist im Teil der Notationen gelistet. Das Prinzip des Topic-Modellings kann mit unterschiedlichen Ansätzen implementiert werden. Beispiele sind: LDA, NNF und LSA. Da diese sich in ihrem Ansatz und Umsetzung unterscheiden, werden im nächsten Abschnitt die bekanntesten dieser Modelle verglichen indem ihre Vorgehens- und Funktionsweise untersucht werden. Die Methode, die in dieser Arbeit genauer beleuchtet wird, ist Latent Dirichlet Allocation. Der dritte Abschnitt wird von der Herkunft, dem Format und der Verarbeitung der Daten der ZVO handeln. Dabei wird der Anwendungsfall und die ZVO genauer erklärt und ein Einblick in die Daten, deren Reinigung und Verarbeitung gegeben. Darauf folgt das Konzept der Lösungsstrategie, um die Aussagekraft und Qualität der Daten für Topic-Modellierung erfolgreich zu untersuchen. Dafür wird der Ablauf des Algorithmus anschaulich dargestellt und die Prozesskette der Analyse im Detail aufgeschlüsselt. Im vierten Abschnitt findet die Implementierung des im dritten Abschnitt erklärten Konzepts statt. Für diese wird eine Methode des Topic-Modellings ausgewählt und in einer vorher definierten und begründeten Programmierumgebung realisiert. Die Umsetzung wird anhand des gelisteten Algorithmus zeilenweise erklärt. Der Output ist in die Arbeit integriert, um dessen zentrale Rolle in der Datenanalyse zu unterstreichen und die Ergebnisse verständlich begründen zu können. Die detaillierte Analyse folgt im nächsten Abschnitt, der sich damit befasst, die Ergebnisse der Implementierung im Anwendungsfall der ZVO zu interpretieren. Dabei wird begründet, wie gut sich die Daten für eine Topic-Modellierung der ZVO Daten eignen und was die Ergebnisse über die Qualität der Daten aussagt. Die Die Arbeit wird abgeschlossen mit der Zusammenfassung der gesamten Analyse und einem Ausblick für die ZVO in Bezug auf Topic-Modellierung. Die gewonnenen Ergebnisse der Datenqualität sollen der ZVO nach der Ergebnisgewinnung zu einer höheren Effizienz in der Datenverarbeitung verhelfen und Aufschluss über Handlungsbedarf und Optimierungspotential in den Datensätzen geben.


\section{Motivation}
Die digitalisierte Welt generiert täglich riesige Mengen an neuen Informationen. Von E-Books, Blogs über Nachrichten-Websites und Magazinen bis hin zu mobilen Anwendungen auf dem Smartphone, immer mehr Menschen verlassen sich auf und richten ihr Leben nach dem Internet aus. Das Zeitalter des Big-Data ermöglicht es Nutzern unlimitiert viele Daten zu generieren und zu sammeln. Die Kapazitäten, die ein Mensch aufbringen kann, um solche Massen an Daten zu organisieren und zu verstehen, sind schon lange übertroffen. Vorallem durch die steigende digitale Kommunikation und stetig sinkenden Speicherplatzkosten, erhöht sich die Menge an zu speichernden Einsen und Nullen. Laut Statista wurden 2018 33 Zettabyte an Daten generiert mit einer prognostizierten Steigerung bis 2025 um 530\% auf 175 Zettabyte. Dieser dramatische Anstieg zeigt die Dringlichkeit für effiziente Algorithmen und Modelle der Datenverarbeitung. Neben der reinen Handhabung solcher Daten steigt aber auch das Bewusstsein, aus diesen Daten Verständnis und Potentiale zu schaffen. Besonders Suchverfahren gewinnen an Bedeutung, wenn in großen, unübersichtlichen Datenmengen bestimmte Informationen gefragt sind. Für die Mehrheit bieten Firmen, wie Google, diese Anwendung an. Zwar kann durch die Keyword-basierten Suche das passenste Dokument gefunden werden, jedoch schlägt die Suchmaschine fehl, wenn nach einer Menge von Dokumente mit einem übergreifenden Thema gefragt ist. Um mehrere Dokumente auf geteilte Themen zu untersuchen, wird das sogenannte Topic-Modelling verwendet. Topic Modeling (dt. Themenmodellierung) beschreibt eine Gruppe von Verfahren, die es ermöglichen, große elektronische Datensammlungen automatisiert zu durchsuchen, organisieren und zu verstehen. Es können Muster innerhalb der Daten entdeckt und Themen extrahiert werden. Dabei stellen Topic Modelle statistische Modelle dar, die Verwendung in der Inferenz abstrakter Topics in unsortierten Datenmengen finden. Topic Modelling gehört zu dem Bereich des Natural Language Processing, also der Verarbeitung natürlciher Sprache. Es verbindet Computerlinguistik, Informatik und Künstliche Intelligenz, um die Potentiale der Sprachverarbeitung mit der heutigen Technik auszuschöpfen. In einer Welt von exponentiell wachsenden Datenmengen finden Methoden des Topic Modeling stetig eine breitere Anwendung. Bereits heute wird Topic Modeling in vielen Bereichen der Wirtschaft, Wissenschaft und Informationstechnologie verwendet. So findet Topic Modeling unter anderem Anwendung bei Zusammenfassungen, Spam Filtern, Internet of Things (IOT), Healthcare, Blockchain, Chatbots, FAQs oder HR. Dies zeigt wie umfangreich das Anwendungsspektrum des Topic Modelings ist. Um semantische Folgerungen aus Datenmengen zu generieren, gibt es verschiedene Ansätze – in dieser Arbeit wird es um das generative Modell ‚Latent Dirichlet Allocation‘ gehen. Dabei werden ähnliche Wörter, die in ähnlichen Kontexten vorkommen in Topics gruppiert. Die Grundlagen des LDA liegen bei der Verallgemeinerung des bereits $1999$ veröffentlichten 'Probabilistic Latent semantic Analysis (PLSA)'. Im Gegensatz zu anderen Machine Learning Methoden im Bereich der Datenverarbeitung, hat Topic Modellierung die Besonderheit, dass ein Dokument nicht nur zu einem Topic zugeordnet werden kann, wie z.B. bei Clusteralgorithmen. Bei der Topic Modellierung wird jedes Dokument durch eine Verteilung an Topics beschreiben, das bedeutet in jedem Dokument sind immer alle Topics zu finden - nur zu einem bestimmten Anteil. Genauso sind in einem Topic immer alle Worte zu einer bestimmten Wahrscheinlichkeit vorhanden. Bei einem Artikel, der zu $90\%$ über Sport und $10\%$ über Politik handelt, werden somit neun mal mehr Wörter bezüglich Sport zu finden sein, als über Politik. Topic Modeling wird den unüberwachten Lernmethoden des Data Minings zugeordnet, also der Extraktion von Muster und Trends in Datenmengen durch die Anwendung statistischer Algorithmen. Das bedeutet, dass die Topics ohne die Einwirkung von manuell erzeugten Labels gefunden werden. Im Lernprozess werden dann Verteilungen basierend auf den bislang vorgenommenen Zuordnungen iterativ angepasst und verbessert - jedoch alles ohne menschliches Zutun. 

\section{Ziel}
Diese Arbeit wird die Theorie des Topic Modeling anhand des Beispiels des Zweckverband Ostholstein (ZVO) in der Praxis implementieren und die bezüglichen Parameter im Sinne der Auswertung bewerten. Der ZVO ist ein Unternehmen, das in Norddeutschland in der Energie-, Entwässerungs-, Internet- und Entsorgungsbranche tätig ist.  erhält jährlich ein große Menge an Kundenanfrage, die in eine oder mehrere der 18 Abteilungen zugeordnet werden müssen. Diese werden momentan händisch an die jeweils zuständige Abteilung weitergeleitet, was sowohl zeitintensiv, als auch fehleranfällig zu Ineffizienz in der Wertschöpfungskette führt. Der Prozess soll zukünftig automatisch durch einen Klassifikationsmechanismus funktionieren. Nach der Implementation eines LDA Algorithmus zur Inferenz verschiedener Abteilungen aus den Kundenanfragen, kann die momentan händische Kategorisierung bewertet werden. Diese Arbeit beschäftigt sich mit der Vorhersage der Qualität des Klassifikators, indem die Qualität der manuell erstellten Kategorien und Kundenanfrage-Gruppen untersucht und mit den Ergebnissen verschiedenen Topic Modellierungen verglichen wird. Das Ergebnis eines Topic Models hängt stark von der Qualität der Daten ab, die sie als Input bekommt. Diese Daten durchlaufen eine Reinigungsphase, bevor sie klassifiziert werden, um sie in eine gut zu verarbeitende Form zu bringen. Ziel dieser Arbeit ist es, Erkenntnisse über die Qualität des Klassifikators zu treffen, in Abhängigkeit zu den verwendeten Daten. Somit wird durch die Nutzung von LDA-Modellen die Qualität der Daten untersucht und Prognosen über einer höheren Qualität der Klassfikiation anhand der Daten gemacht. 

\chapter{Grundlagen}%%%%%%%%%%%%%%%%%%%%%%%%%%%%%%%%%%%%%%%%%%%%%%%%%%%%%%%%%%%%GRUNDLADEN


\section{Notation}

\begin{itemize}
	\item $\mathcal{K}$ ist die Anzahl der Topics in einem Topic-Modell $\mathcal{M}$
	\item Ein Model $\mathcal{M}$ repräsentiert den Korpus $\mathcal{D}$
	\item Eine Menge von Dokumenten ist ein Korpus $\mathcal{D}$
	\item Ein Dokument $\mathcal{d}$ eines Korpus ist eine Menge von $\mathcal{N}$-vielen 
	Worten 
\end{itemize}


\section{Modellvergleich}

\subsection{Latent Semantic Analysis}
Ein anderes verbreitetes Verfahren ist das „Latent Semantic Analysis“ (LSA), welches auf das Finden von sogenannten Hauptkomponenten in Dokumenten abzielt. Dadurch können sowohl ähnliche Wörter gefunden, als auch Textbereiche, die inhaltliche Überschneidungen mit einem bestimmten Begriff haben, aber das Wort selber nicht enthalten, gefunden werden. Die Methode basiert auf dem Prinzip der Singulärwertzerlegung(SVD). Als Ausgangslage wird aus einer Textsammlung eine Term-Dokument-Matrix erstellt. Diese Matrix wird in der SVD als Produkt von drei Matrizen dargestellt, von denen die mittlere eine Diagonalmatrix darstellt. Die Werte auf der Diagonalen lassen daraus die Topics der Textmenge ablesen. Auf das SVD Verfahren selbst hat der Entwickler wenig Einfluss. Um Rauschen zu verhindern, kann jedoch die Anfangsmatrix mithilfe der term-frequency und inverse-document-frequency (tf-idf) verbessert werden, was sich auf das Gesamtergebnis auswirkt. LSA stellt sich als ein attraktives Verfahren heraus, da es Synonyme besser erkennen kann, als LDA und wird heutzutage unter anderem intensiv in dem Bereich des Digital Marketings genutzt.

\subsection{Probabilistic Latent Semantic Analysis}
Im Gegensatz zu LSA, was Singulärwertzerlegung für die Topic-Modellierung nutzt, verfolgt Probabilistic Latent Semantic Analysis (PLSA) einen anderen Ansatz. Die Kernidee ist, ein Wahrscheinlichkeitsmodell zu finden, das ein Dokument erzeugt, dass zu der Dokument-Term Matrix passt. PLSA hat als erstes Modell dem Konzept des Topic-Modellings Wahrscheinlichkeiten hinzugefügt, sodass sich der Fall, dass Dokument $d$ über Topic $z$ handelt, mit einer Wahrscheinlichkeit von $P(z\|d)$ beschreiben lässt. Dazu berechnet $P(w\|z)$, wie wahrscheinlich es ist, dass das Wort $w$ aus einem gegebenen Dokument $d$ zu Topic $z$ gehört. Somit führt PLSA die Gesamtwahrscheinlich ein, dass ein Wort in einem gegebenen Dokument zu finden ist: $P(D,W) = P(D) \sum_{Z}{P(Z\|D)P(W\|Z)}$. Die Gleichung verbindet die Wahrscheinlichkeit ein bestimmtes Dokument zu erhalten mit der eines Wortes in diesem bedingt durch die Topic-Verteilung in diesem Dokument. Zusammenfassend ist PLSA sehr ähnlich zu LSA, auch wenn es einen anderen Ansatz nutzt, mit einer additiven pobabilistischen Komponente. 

\subsection{Latent Dirichlet Allocation}
Topic Modeling besteht aus vielen Methoden, die meist verbreitete ist die „Latent Dirichlet Allocation (LDA)“.  Dieses Verfahren geht ursprünglich auf PLSA zurück, ergänzt durch zwei Dirichlet-Verteilungen. LDA liegt ein generierender Prozess zugrunde, den zwei Dirichlet Verteilungen maßgeblich beeinflussen: die Dokument-Topic-Verteilung, die die Ausprägungen verschiedener Topics in einem Dokument beschreibt, und die Topic-Word-Verteilung, die die Wahrscheinlichkeit beschreibt, dass ein bestimmtes Wort in einer gewissen Regularität in einem Themenbereich vorkommt. Dabei geht man davon aus, dass ein Dokument eine Verteilung von Topics ist, während ein Topic als eine Verteilung über Wörter betrachtet wird. 
Die Wahrscheinlichkeit, dass ein bestimmtes Dokument generiert wird, ist das Produkt der Wahrscheinlichkeiten der beiden Verteilungen mit den Wahrscheinlichkeiten zweier multinomialen Verteilungen, die erst zufällig Topics, wie in der Dirichlet-Verteilung definiert, auswählen und aus diesen dann, mithilfe der zweiten Dirichlet-Verteilung, Wörter aus diesen Topics herleiten, wodurch das Enddokument entsteht. Das Enddokument wird höchstwahrscheinlich stark von dem gegebenen Dokument abweichen, jedoch kann durch anpassen der Dirichlet-Verteilungen ein Optimierungsproblem formuliert werden, nach dem die Dirichlet-Verteilungen gesucht werden, die ein möglichst ähnliches Dokument generieren.

\subsection{Non-Negative Matrix Factorization}
Ein weiteres Verfahren, das auch mit Matrizen funktioniert, wird „Non-Negative Matrix Factorization“ (NMF) genannt. Dabei wird eine Matrix, die Wörter auf Dokumente abbildet, in zwei Teilmatrizen faktorisiert. Dabei müssen alle Werte positiv sein. Faktorisierung bezeichnet die Darstellung einer Matrix durch die Multiplikation zweier anderer Matrizen. Dabei ist es bei NNM nicht immer möglich die originale Matrix mit der Multiplikation zu regenerieren, deshalb wird diese bestmöglich approximiert. 

Die erste Teilmatrix stellt die Topics in Dokumenten, die zweite die Wörter in Topics dar. Dadurch kann Speicherplatz gespart, und Themen aufgedeckt werden. Das Verfahren beginnt mit zwei möglichen faktorisierten Matrizen und verbessert sich durch die Errorfunktion iterativ, bis das Ergebnis gut genug ist. Dabei werden die errechneten Werte mit der gegebenen Matrix verglichen und angepasst.


\section{Grundlagen der Latent Dirichlet Allocation }
Latent Dirichlet Allocation ist ein grundlegendes und bekanntes Verfahren aus der Natürlichen Sprachverarbeitung. Begründet wird dies unter anderem auf der Komplexität der damals bestehenden Techniken der Textverarbeitung. So waren Clusteralgorithmen zu starr in ihrem Anwendungsumfeld, während Dimensionsreduktionen, wie die Hauptkomponentenanalyse Ergebnisse lieferte, die sehr schwer zu interpretieren waren. Das Prinzip des Topic Modeling basiert auf einer Menge an Dokumenten, die den Korpus darstellen. Dabei werden bei LDA alle Dokumente als Menge von Wörtern angenommen, die als Bag of Words modelliert sind. Dabei hat weder die Reihenfolge, noch die Groß- und Kleinschreibung Einfluss auf das Ergebnis. Die Themen werden  allein an der Vorkommenswahrscheinlichkeit der Wörter ohne Reihenfolgen- oder Kontextinformationen erkannt. Durch die Reduktion der Dimension wird die Effizienz gesteigert. Somit wird also jedes Dokument durch eine Verteilung der enthaltenen Wörter repräsentiert.\\

Bezüglich der Namensgeben, steht "Latent" für alles, was wir im Vorhinein nicht kennen. Im Fall LDA handelt es sich um die Themen, die in einem Dokument zu einem bestimmten Teil vertreten sind. "Dirichlet" beschreibt eine Verteilung von Verteilungen. Dies ist vergleichbar mit einem Würfel, bei dem regulierbar ist, wie gleichmäßig die Zahlen gewürfelt werden. Dabei ist der Würfel eine Verteilung und die Aufteilung der Gleichmäßigkeit auch. Beim Topic Modeling bedeutet Dirichlet eine Verteilung von Topics in Dokumenten und eine Verteilung von Wörtern in Topics. Die "Allocation" weist  mithilfe der errechneten Dirichlet-Verteilungen Topics Wörter und Dokumenten Topics zu. Eine Besonderheit bei der Themenerkennung mit LDA ist, dass die Anzahl der gesuchten Themen \mathcal{K} vorgegeben werden muss. Oft ist diese vorher jedoch nicht bekannt und muss über Hilfsverfahren, wie der Perplexitätsberechnung ermittelt werden. Die Funktionsweise von LDA ist über folgende graphische Abbildung beschrieben: \\

\begin{figure}[h]
\centering
\begin{tikzpicture}
	\draw (0,0) [very thick] rectangle (9,3) 
	\draw [very thick](6,3.5) rectangle (9,5.5)
	\draw [very thick](3,0.25) rectangle (8.75,2.75)
	\draw (1.5,1.5) circle [very thick, radius = 0.8]  node {$\theta$};
	\draw (4.5,1.5)  circle [very thick, radius = 0.8] node {$\mathcal{Z}$};
	\draw (7.5,1.5) circle [fill=black, ultra thick, radius = 0.8, accepting] node {$W$};
	\draw (-1.5,1.5) circle [very thick, radius = 0.8] node {$\alpha$};
	\draw (7.5,4.5) circle  [very thick, radius = 0.8] node {$\beta$};
	\draw (4.5,4.5) circle  [very thick, radius = 0.8] node {$\phi$};
	
	\draw [->,ultra thick] (5.3,4.5) -- (6.7,4.5)
	\draw [->,ultra thick] (7.5,3.7) -- (7.5,2.3)
	\draw [->,ultra thick] (5.3,1.5) -- (6.7,1.5)
	\draw [->,ultra thick] (2.3,1.5) -- (3.7,1.5)
	\draw [->,ultra thick] (-0.7,1.5) -- (0.7,1.5)
\end{tikzpicture}
\caption{Graphische Darstellung von LDA}
\end{figure}

Dabei beschreibt \mathcal{W} als einzige nicht verborgene Variable eines von \mathcal{N} Wörtern des Dokuments. Das Wort ist semantisch einem Thema \mathcal{Z} zugeordnet. Das Thema wiederum hängt von der Themen-Verteilung \mathcal{theta} des Dokuments ab, das als ein Elment der \mathcal{M} vorliegenden Dokumente betrachtet wird. Neben dem Thema, wird jedes Wort auch von der jeweilgen Thema-Wort-Verteilung \mathcal{\phi} der \mathcal{K} Themen beeinflusst. Das Modell und dessen Verteilungen kann durch die Parameter $\alpha$ und $\beta$ angepasst werden. $\alpha$ kann bestimmt die Intensität der Dokument-Themen-Verteilung, während $\beta$ die der Topic-Wort-Verteilung beeinflusst. Bei einem großen $\alpha$ ist die Verteilung der Topics in einem Dokument ähnlicher. Zusätzlich werden bei LDA zwei Bedingungen verfolgt, die von den beiden Parametern beeinflusst werden können. Erstens strebt man für alle Wort eines bestimmten Dokuments so wenig zugeordnete Themen an, wie möglich. Zweitens soll ein Thema über so wenig relevante Wörter wie möglich verfügen. Die beiden Ziele stehen in einer Wechselbeziehung zueinander, da eine minimale Anzahl an vertretenen Topics in einem Dokument zu maximal vielen Wörtern in diesen Topics führt. Die minimale Anzahl an Topics wäre erreicht, wenn man alle Wörter eines Dokuments einem Thema zuweist. Dadurch verfügt das Topic jedoch über alle Wörter des Dokuments. $\alpha$ befindet sich in dem Bereicht $[0,1]$ mit sinnvollen Werten zwischen $[0.01, 0.1]$, während $\beta =0.01$ durchschnittlich die besten Ergebnisse liefert. Große Werte führen zu einer Gleichverteilung, die wiederum eine Verschlechterung der Perplexität bedeutet. Somit bietet die Perplexität ein Mittel, um $\alpha$ und $\beta$ optimal für die individuelle Anwendung zu finden. \\

Bei LDA werden zwei Verteilungen aus den Dokumenten $d \in \mathcal{D}$ und $k \in \mathcal{K}$gelernt: die Dokument-Topic-Verteilung $\theta$ und die Topic-Wort-Verteilung $\phi$. Dabei gibt die Dokument-Topic-Verteilung an, mit welcher Wahrscheinlichkeit das Dokument zu jedem Themen gehört. Die Topic-Wort-Verteilung berechnet die Wahrscheinlichkeit, dass ein Wort einem Thema angehört. $\mathcal{M} = LDA(\mathcal{D})$ beschreibt ein LDA Modell, das auf der Dokumentenmenge/Korpus $\mathcal{D}$ trainiert wurde.


\subsection{Der generative Prozess} 
Bei der Klassifikatorenerstellung gibt es zwei unterschiedliche Herangehensweisen: den deskriptiven und den generativen Ansatz. Bei der deskriptiven, oder auch beschreibenden, Statistik geht es um die sinnvolle und übersichtliche Darstellung empirischer Daten durch zum Beispiel Tabellen oder Kennzahlen. Betrachtet man zwei Variablen, ein bekanntes $\mathcal{X}$ und eine gesuchte Variable $\mathcal{Y}$, dann wird im deskriptiven Modell die bedingte Wahrscheinlichkeit von $\mathcal{Y}$ bedingt durch $\mathcal{X}$ betrachtet. Im Gegensatz dazu ist bei generativen Modellen die Wahrscheinlichkeit von $\mathcal{X}$ und $\mathcal{Y}$ gemeinsam relevant. Bei generativen Modellen besteht die Möglichkeit, Ausgabeinstanzen zu erstellen. Beispiele dafür sind das "Hidden Markov Modell", Bayes Netze oder das "Gaussian mixure model". Bei LDA handelt es sich um einen generativen Algorithmus. Somit können theoretisch im Fall einer neuen Dokumentzuordnung zu Abteilungen für jede Abteilung zufällig Dokumente generiert werden und diese mit dem neuen Dokument verglichen werden. Der Algorithmus hinter LDA generiert neue Dokumente mithilfe von Dirichletverteilungen $Dir(\gamma)$ und Multinomialverteilungen $Multinom(\delta)$. Dabei sorgen Multinomialverteilungen dafür, dass ein Dokument Teil mehrerer Topics sein kann. Die Verteilungen $\theta$ und $\phi$ werden errechnet, indem iterativ neue Dokumente über andere Verteilung generiert werden, bis das generierte Dokument die Anforderungen befriedigt, dann können die Verteilungen abgelesen werden, mit denen das Dokument erstellt wurde. Der Prozess verläuft folgendermaßen: 
\begin{enumerate}
	\item Wähle ein $\theta$ als $Dir(\alpha)$
	\item Wähle ein $\phi$ als $Dir(\beta)$
	\item Für jedes Wort $w$ and Stelle $i = 1,...,N$ im Dokument $d$: 
	\begin{enumerate}
		\item Wähle ein Thema $z_{d,i}$ als $Multinom(\theta_d)$
		\item Wähle ein Wort $w_{d,i}$ als $Multinom(\phi_z_{d,i})$
	\end{enumerate}
\end{enumerate}

Somit kann der Algorithmus nun neue Dokumente erstellen und das Ergebnis durch die Parameter, wie Alpha und Beta, anpassen, bis das Ergebnis ähnlich genug zu dem Anfangsdokument ist. Dann ist die Verteilung der Themen in diesem Dokument bekannt. Bei der Anwendung von LDA für praktische Problemstellungen, geht LDA das Prinzip rückwärts durch, d.h. für bestehende Gruppen an Dokumenten werden Verteilungen gesucht, durch die das Dokument generiert hätte werden können. \\

\begin{equation}
P(w,z,\theta, \phi, \alpha, \beta) = \prod P(\theta, \alpha) \cdot \prod P(\alpha, \beta) \cdot \prod P(z,\tehta) \cdot \prod P(w \mid \phi) 
\end{equation}
\\
Die Formel beschreibt die totale Wahrscheinlichkeit des LDA Modells. Sie setzt sich zusammen aus den Produkten der Dirichlet Verteilung der Topics und der Wörter zusammen mit den multinomialen Verteilungen der Topics und Wörter. Die Schwierigkeit des Algorithmus besteht in der Berechnung der $\theta$-Verteilung der gegebenen Dokumente für latente Variablen. Dies lässt sich durch folgende Wahrscheinlichkeitsverteilung ausdrücken: 

\begin{center}
\begin{equation}
P(\theta, z \mid w, \alpha, \beta) = \frac{P(\theta, z, w \mid \alpha, \beta)}{P(w \mid \alpha, \beta)}
\end{equation}
\end{center}\\
\\
\\
Die Formel berechnet die Wahrscheinlichkeit der Verteilung unter einem bestimmten Topic gegeben der $\alpha$ und $\beta$ Parameter und dem bekannten Wort. Die Wahrscheinlichkeit kann nicht exakt bestimmt werden, weshalb Verfahren wie Gibbs Sampling diese approximieren.\\

\subsection{Alpha und Beta}
Die Dirichlet Verteilungen werden durch die beiden Parameter $\alpha$ und $\beta$ bestimmt. Diese formulieren die mathematische Bedeutung der beiden Ziele von LDA:

\begin{enumerate}
	\item Ein Dokument wird so wenigen Themen wie möglich zugewiesen ($\alpha$)
	\item Jedes Thema hat so wenig relevante Wörter wie möglich ($\beta$)
\end{enumerate}

Dabei kann 1) erreicht werden, wenn alle Worte eine Topic wären, was jedoch nicht mit 2) übereinstimmen würde. Für ein erfülltes 2) gibt es nicht die minimale Anzahl an Topics. Die Funktionsweise von verschiedenen $alpha$-Werten zeigen folgende Abbildungen: 
\\
\begin{figure}[h]
\begin{center}
\includegraphics[scale=0.3]{lda_alpha001.png}
\end{center}
\caption{Ein kleines Alpha} 
\end{figure}\\
\begin{figure}[h]
\begin{center}
\includegraphics[scale=0.3]{lda_alpha1.png}
\caption{Ein. großes Alpha}
\end{center}
\end{figure}\\
In der ersten Abbildung führt ein kleiner $\alpha = 0.01$ zu einer sehr eindeutigen Themenvertieilung. Bei der unteren Abbildung hingegen haben wir ein $\alpha = 1$, was eine gleichmäßigere Verteilung zur Folge hat. Der Trade-off zwischen den beiden Zielen ist der Grund für das funktionieren von LDA. Dadurch wird eine

%\begin{center}
%\includegraphics{img1.png}
%\end{center}


\chapter{Konzept} %%%%%%%%%%%%%%%%%%%%%%%%%%%%%%%%%%%%%%%%%%%%%%%%%%%%%%%%%%%%%%%%%KONZEPT

\section{Daten}
Die Daten liegen in folgendem Format vor: 

%\begin{center}
%\includegraphics{daten1}
%\end{center}

\begin{center}
\begin{figure}[h]
\begin{tabular}{ccccccccccc}

\hline
\hline
& filname & subject-message & Abt0 & Abt1 & Abt2 & Abt3 & ... & Abt15 & Abt16 & Abt17\\
\hline
0 & FILE0& content0 &0&0&0&1&...&0&0&0\\
1& FILE1 & content1 &0 &0 &0 &0 &... &1&0&0\\
2& FILE2 & content2 & 1 &0 &0 &0 &...&0 &0&0 \\
3& FILE3 & content3 & 0 & 0  &0 &1 &...&0 &1&0 \\
4& FILE4 & content4 & 0 & 1 &0 &0 &...&0 &0&1 \\
...& ... & ... & ... & ...  &... &... &...&... & ...&... \\
133044& FILE133044 & content133044 & 0 & 0  &1 & 0&...&0 &0&0 \\
\hline
\hline
\end{tabuar}
\caption{Daten TODO...}
\end{figure}
\end{center}

Relevant für die Auswertung sind die subject-message und die jeweilige Abteilung. Die Tabelle verfügt über eine Matrix mit 18 Abteilungen, von denen pro subject-message eine oder mehrere mit einer 1 versehen ist bzw. sind. Dies beschreibt die Abteilung bzw. Abteilungen, der bzw. denen diese Anfrage manuell zugeordnet wurde. Die Daten in subject-message sind bereits bereinigt, also liegen wie in diesem künstlichen Beispiel vor:\\

%\begin{center}
%\includegraphics{daten2}
%\end{center}

\begin{figure}[h]
\begin{lstlisting}
OUTPUT: 
'wasser verbraucht amt deutschland ablesung zaehlen strom voll ort luebeck art straße messung verband nummer platz markieren wechsel lieferant stelle verbrauch kunde kunden anrede mann sommer beschwerde schrift allgemein kommunikation datenmanagement fern'
\end{lstlisting}
\caption{Ein ausgedachtet Datum aus der Tabelle der ZVO}
\end{figure}\\

Um die Einträge in eine computer-lesbare Form zu verwandeln, muss ein Dictionary erstellt werden, dass alle Wörter auf eine Anzahl ihrer Vorkommen abbildet. Dafür müssen die Wörter als alleinige Listeneinträge einlesbar sein: \\

%\begin{center}
%\includegraphics{daten3}
%\end{center}

\begin{figure}[h]
\begin{lstlisting}
OUTPUT split: 
['wasser', 'verbraucht', 'amt', 'deutschland', 'ablesung', 'zaehlen', 'strom', 'voll', 'ort', 'luebeck', 'art', 'straße', 'messung', 'verband', 'nummer', 'platz', 'markieren', 'wechsel', 'lieferant', 'stelle', 'verbrauch', 'kunde', 'kunden', 'anrede', 'mann', 'sommer', 'beschwerde', 'schrift', 'allgemein', 'kommunikation', 'datenmanagement', 'fern']
\end{lstlisting}
\caption{Ein gesplittetes Datum vorbereitet für das Dictionary}
\end{figure}\\


\subsection{Datenreinigung}
Bevor eine Themenmodellierung auf Daten durchgeführt werden kann, müssen die Daten einem Prozess unterzogen werden. Dieser beginnt mit der Datenaquise, also der Akquirierung bestimmter relevanter Daten. Im Falle der ZVO bedeutet dies, dass es genügend Kundenanfragen gibt, die verarbeitet werden können. Wenn diese Daten bestehen, werden sie auf die relevanten Wörter reduziert, aus denen eine bedeutsame Inferenz von Informationen möglich ist, sodass unter anderem die sogeneannten „Stop-Words“, also eine Menge von Verbindungswörtern entfernt werden. Ein anderer Schritt der Datenreinigung ist das Transponieren aller Wörter in kleine Buchstaben, um eine Einheitlichkeit zu erlangen, da das Bag of Words Modell keine Reihenfolge mehr beachtet und somit große Satzanfänge irrelevant werden. Wenn die Daten in der gewünschten Form vorliegen, beginnt der Schritt des Featureengineerings. Für einen Computer sind Wörter nicht so leicht zu verarbeiten, wie Zahlen, weshalb in diesem Schritt eine Quantisierung der Wörter und Überführung dieser in eine zahlenbasierte Form vorgenommen wird. Dies kann zum Beispiel in Form eines Bag-of-Words Modells, Dictionary oder TF-IDF, also einer relativen Vorkommensauflistung verschiedner Wörter über Dokumente umgesetzt werden. Nachdem die Daten in eine für den Computer kompatiblen Form gebracht wurden, kann das Themenmodell entwickelt werden. 

\subsection{Feature Engineering}
Das alleinige Reinigen der Daten reicht nicht aus, um das LDA Topic Modell auf diesen zu generieren. Für eine maschinelle Verarbeitung sind die Wörter in der Form nicht adressierbar. Bei LDA ist ein elementarer Bestandteil, wie oft ein Wort vorkommt. Das bedeutet im Feature Engineering müssen die Daten in ein Format übersetzt werden, das für jedes Wort ohne Duplikate die Anzahl mit einer Wort-ID referenziert. Die Auflistung der Wörter zusammen mit ihrer Vorkommensanzahl und laufenden Indexnummer kann als Input für ein LDA Modell verwendet werden. Dies nennt sich das Wörterbuch (o.a. Dictionary). 

\textbf{TF-IDF und Word Embedding}
Die Art, wie Wörter repräsentiert werden kann sich unterschieden. Das Dictionary wird bei beiden kreiert. TODO ERKLÄRE TF-IDF UND WORD EMBEDDIN


\section{Anwendungsfall ZVO}
Bei der ZVO sollen jährlich händisch aufgenommene Anfragen in Zukunft maschinell klassifiziert werden. Für die korrekte Klassifikation ist die Qualität der vorliegenden Daten immens wichtig. Die Qualität der ZVO Daten werden in dieser Arbeit untersucht. Dafür wird ein LDA Modell generiert, das als Datengrundlage alle verfügbaren ZVO Daten verwendet. Die Qualitätsuntersuchung der Daten wird durch zwei Methoden durchgeführt. Dies zeigt das folgende Prozessablaufdiagramm:  

% FLOW CHART %%%%%%%
\tikzstyle{rect} = [rectangle, text centered, draw =black]
\tikzstyle{arrow} = [thick,->,>=stealth]

\begin{figure}[H]
\begin{center}
\begin{tikzpicture}[node distance = 2cm]

\node (mails) [rect] {Datenerhebung durch Mailanfragen};
\node (zuordnung) [rect, below of = mails] {manuelle Zuordnung Daten zu Abteilungen};
\node (Datenreinigung) [rect, below of = zuordnung] {Datenreinigung};
\node (Feature) [rect, below of = Datenreinigung] {Feature Engineering};
\node (LDA) [rect, below of= Feature] {LDA Modell erstellen};
\node (Schluesselwoerter) [rect, below right of = LDA, node distance = 3cm]{Schlüsselwörter finden};
\node (Topics) [rect, below left of = LDA, node distance = 3cm] {Topics finden};
\node (Matrix) [rect, below of = Schluesselwoerter] {Matrix erstellen};
\node (Ueberschneidungen) [rect, below of = Matrix] {Überschneidungen analysieren};
\node (Topicsana)[rect, below of=Topics]{Topics und Keywords analysieren};
\node (d1)[rect, right of=Datenreinigung]{Stopwords};
\node (d2)[rect, right of=d1]{Lemmatizing};
\node (f1)[rect, right of=Feature]{Bag of Words};

\draw [arrow] (mails) -- (zuordnung);
\draw [arrow] (zuordnung) -- (Datenreinigung);
\draw [arrow] (Datenreinigung) -- (Feature);
\draw [arrow] (Feature) -- (LDA);
\draw [arrow] (LDA) -- (Schluesselwoerter);
\draw [arrow] (LDA) -- (Topics);
\draw [arrow] (Topics) -- (Topicsana);
\draw [arrow] (Schluesselwoerter) -- (Matrix);
\draw [arrow] (Matrix) -- (Ueberschneidungen);
\draw [arrow] (Matrix) -- (Ueberschneidungen);

\end{tikzpicture}
\caption{Der Ablauf .. TODO}
\label{fig:flowchart}
\end{center}
\end{figure}

Nach TEST \ref{fig:flowchart}} der Erhebung der Daten können die einzelnen Anfragen manuell in die vorgegebenen Abteilungsgruppen eingeteilt werden. Dort wird die Datenreinigung, wie in 3.2 beschrieben vorgenommen. Sind die Daten bereinigt, kann das Feature Engineering beginnen, wonach die Daten für den Computer verständlich formatiert sind. Die Erstellung des LDA Modells wird vorgenommen, sobald der Korpus aus den feature engineerten Daten erfolgt ist. Die Modellerstellung ist durch folgenden Pseudocode beschrieben: \\
\\

\begin{figure}[h]
%\begin{algorithm}
\begin{algorithmic}
\STATE data $\leftarrow " "$
\FOR{d in Anzahl Dokumente} $data \leftarrow data + str(d)$\ENDFOR 
\STATE Teile data in einzelne items einer Liste auf
\STATE Erstelle ein Dictionary aus der Liste
\STATE Wandle ID aus Dictionary in Wörter um
\STATE Erstelle den Korpus
\STATE Erstelle das Modell
\STATE Gibt die Topic-Wort-Verteilungen für alle Topics aus
\end{algorithmic}\captionsetup{labelfont={sc,bf}, labelsep=newline}
\caption{Pseudoalgorithmus für die Prozesskette}
%\end{algorithm}
%\caption{Pseudoalgorithmus ... TODO}
\end{figure}
\\
\\

Ist das Modell generiert und die Topics auslesbar, können die Daten evaluiert werden. Dazu werden zwei Ansätze verfolgt. Zum Einen werden Topics und zugehörigen Schlüsselwörter untersucht, um u.a. Aufschluss über mögliche Verbesserungspotentiale bei der Datenreinigung oder dem Feature Engineering feszustellen. Als zweites werden die vom Modell erfassten Gruppen betrachtet und den gegebenen manuell klassifizierten Abteilungen zugeordnet. Die folgenden Punkte beschreiben den Prozess im Detail: \\

\begin{enumerate}
\item \textbf{Gruppen und Wörter finden}\\
Ein LDA Modell besteht aus zwei Verteilungen, die die zugrundeliegenden Daten semantisch darstellbar machen: die Dokument-Topic-Verteilung und die Topic-Wort-Verteilung. Als Ausgabe des Modells ist also zu erkennen, welche Topics die Dokumentmenge durchschnittlich hauptsächlich beschreiben und welche Wörter in den Topics jeweils dominant vorkommen. Das Modell kann Topics nicht inhaltlich bennen, sondern nur die Verteilungen darstellen. Somit ist nicht eindeutig, welches Topic welche Abteilung der ZVO darstellt. Dafür betrachten wir die Topic-Wort-Verteilungen und schließen von dieser auf die Qualität der Daten. Ist über die dominanten Wörter in einem Topic zu erkennen, welche Abteilung dieser repräsentiert, scheint das Modell und die Daten gut genug zu sein, um die Daten zu klassifizieren. Sollte die Abteilung nicht an den Wörtern ablesbar sein, sind die Daten nicht optimal für eine Klassifikation geeignet. 

\item \textbf{Zuordnung Abteilung zu Topic}
Das LDA Modell clustert die Daten in 18 Topics. Diese Topics sollten im optimalen Fall sehr ähnlich zu den händisch klassifizierten Abteilungen sein. Ist dies nicht der Fall, kann man auf eine schlechte Klassifikation schließen. Dies kann durch eine schlechte Qualität der Daten als auch durch eine ineffiziente manuelle Einteilung der Topics bedingt sein. Die Fähigkeit, Topics auf Abteilungen zu mappen, gibt Aufschluss über die Qualität der Daten. Für die Zuordnung werden zwei Matrizen verwendet: \lstinline{gruppen_LABEL} und \lstinline{gruppen_LDA}. Die erste Liste sortiert alle Dokumentindizes als Listenelemente in die jeweilige Zeile der Matrix, sodass der Index eines händisch in Abteilung 3 eingeordneten Dokumentes in \lstinline{gruppen_LABEL[4]} zu finden ist. Die Matrix \lstinline{gruppen_LDA} beinhaltet alle Indizes der Dokumente, die vom Modell klassifiziert wurden, in gleicher Struktur. Dafür wird für jedes Dokument, das den Korpus ausmacht, eine dokumentseigene Dokument-Topic-Verteilung errechnet. Das Topic, für das das Dokument am wahrscheinlichsten ist, bestimmt, welcher Teilliste der Dokumentindex angehängt wird. Beide Matrizen verfügen nun über die Indizes der Dokumente in den jeweiligen Topics bzw. Abteilungen und können anhand der einzigartigen Indizes auf Überschneidungen geprüft werden. Die Anzahl der Überschneidungen werden in einer Matrix gespeichert, die jedes Element von \lstinline{gruppen_LDA} auf jedes Element von \lstinline{gruppen_LABEL} abbildet und deren Überschneidung zählt. Eine optimale Zuordnung von Topic auf Abteilung ist möglich, wenn jede Zeile ein Maximum in einer Spalte hat, die nicht auch das Maximum einer anderen Zeile enthält. Ist dies jedoch nicht der Fall, sind die Daten in der aktuellen Form nicht optimal für die Klassifizierung. 
\end{enumerate}

TODO Welcher Algorithmus, wie implementiert? \\
TODO Abbildung mit flow chart\\
TODO pseudocode von topic model


\chapter{Implementierung}%%%%%%%%%%%%%%%%%%%%%%%%%%%%%%%%%%%%%%%%%%%%%%%%%%IMPLEMENTIERUNG

\section{Topic Modeling Methode}
Zur Untersuchung der Qualität der ZVO-Daten wird in dieser Arbeit die LDA Methode verwendet. Dabei wird nur der Text als bekannt angenommen. Weder die Meta-Daten, noch die Anmerkungen oder Labels sind zu Beginn bekannt. Als grundlegendes Topic Modellierungsverfahren findet es Verwendung in einem breiten Anwendungsspektrum. Durch die Bekanntheit von LDA sind bereits viele Pakete und Bibliotheken in Programmierumgebungen vorzufinden und einfach zu implementieren. Seit LDAs Veröffentlichung in 2000 wurde eine umfassende und detailreiche Dokumentation entwickelt, die neben vielen Forenbeiträgen, die Arbeit mit LDA stark erleichtern. Zusätzlich hat LDA in diesem Anwendungsfall den Vorteil, dass es nicht wie zum Beispiel LSA direkt Dokumentähnlichkeiten ausgibt, sondern das Ergebnis in Form einer Matrix dartellt, die Wörter auf Dokumente abbildet. Damit ergeben sich als Werte der Matrix die Topics, denen die Wörter jeweils angehören. Bei LDA Modellen ist die Anzahl der Topics ein individueller Input, durch den sich das Ergebnis schwerwiegend verändern kann. Die Werte in der Matrix können somit von $0$ bis zu der individuellen Anzahl der Topics reichen. Die optimale Anzahl an Topics ist grundsätzlich ein nicht einfaches Problem bei Anwendungen. Im Fall der ZVO werden als Anzahl der Topics 18  gewählt, da dies die Anzahl der bereits erstellten Abteilungen ist. 

\section{Toolauswahl}
Bekannte Frameworks zum Topic-Modelling implementieren grundsätzlich ähnliche Algorithmen. Betrachtet wurden in dieser Arbeit Mallet, Gensim und Sci-kit Learn.  Wichtige Schritte sind die Vorbereitung der Daten, die Implementierung, die Auswertung und die Visualisierung. Als Bibliothek wird in dieser Arbeit Gensim verwendet, die für die Verarbeitung von unstrukturierten Daten und Anwendung von unüberwachten Algorithmen bekannt ist. Algorithmen, wie \lstinline{word2vec}, \lstinline{LSI} oder \lstinline{LDA} entdecken automatisch Strukturen durch das Prüfen von gemeinsam auftretenden Mustern im Korpus der Trainingsdaten. Viele bekannte Algorithmen sind bereits in diesem Framework implementiert, was die Umsetzung der Topic-Modellierung vereinfacht. Gensim erlangte in der Vergangenheit Bekanntheit durch seine hochoptimierten Implementationen bekannter Algorithmen und der Schnelligkeit und Verlässlichkeit, mit der diese ausgeführt wurden. Außerdem wird Gensim in Python verfasst, was sich sehr gut für Probleme im Bereich der Data Science eignet. 
%\subsection{Gibbs Sampling}
%Gibbs Sampling ist eine Technik, um ein LDA Modell zu trainieren. Für das menschliche Auge kann ein Wort eines Dokuments intuitiv einem Thema zugordnet werden. Für den Computer ist dies schwer, da er die Semantik des Wortes nicht kennt. Mit Gibbs Sampling werden Wörter iterativ Topics zugewiesen basierend auf allen restlichen Wörtern und Dokumenten. Dabei ist zum Einen relevant, was die führende Topic in dem jeweiligen Dokument ist und zum Anderen welcher Topic die gleichen Wörter im gesamten Korpus bereits am häufigsten zugeordnet wurden. Die beiden Beobachtungen werden multipliziert, um die Kombination mit der maximalen Wahrscheinlichkeit zu finden. Um zu verhindern, dass ein Wort einem Topic nicht zugewiesen werden kann, das sich nicht in dem bestimmten Dokument befindet, wird der Anzahl der jeweils vorkommenden Themen im Dokument $\alpha$ und der Anzahl der belegten Themen des gleichen Wörter in anderen Dokumenten $\beta$ addiert.  Wurde ein Wort einem Topic zugewiesen wird das nächste Wort betrachtet. Nun ist zu beachten, dass sich die Verteilungen zur Berechnung jedes mal ändern. Somit haben wir einen iterativen Lernfortschritt bei der Topicverteilung.


%\subsection{Hellinger Distanz}
%In der Wahrscheinlichkeitsrechnung kann die Hellinger Distanz als Maß genutzt werden, um die Ähnlichkeit zweier Verteilungen zu berechnen. Für diese Anwendung bietet sich die Hellinger Distanz an, da sie oft im Kontext von Modellen und Topics verwendet wird. Zusätzlich lässt sie sich gut in Gensim Implementationen einbauen. Neben der Verteilungsvergleich kann auch eine Hellingermatrix erstellt werden, die jedes Topic des einen Modells mit jedem Topic eines zweiten Modells vergleicht. Somit wird ist es möglich die Ähnlichkeit zwischen zwei Modellen festzustellen. Formal errechnet sich die Distanz zwischen zwei diskreten Verteilungen $X$ und $Y$ aus folgender Gleichung: 

%\begin{center}
%\begin{equation}
%	H(X,Y) = \frac{1}{\sqrt{2}} \sqrt{\sum^k_{i=1}(\sqrt{x_i}-\sqrt{y_i})^2}
%\end{equation}
%\end{center}

%n diesem Abschnitt werden verschiedene Möglichkeiten implementiert und analysiert, wie LDA im Sinne der ZVO genutzt werden kann. Dabei ist die Zielfrage, wie am besten für ein unbekanntes Dokument die bestimmte Abteilung gefunden werden kann. Dafür müssen Korpora bestehen, die bereits durch Verteilung definiert sind, sodass die Dokument-Themen Verteilung für das neues Dokument inferiert werden kann. Wenn die Themenverteilung des neuen Dokuments gegeben ist, kann über Vergleiche der Verteilungen mit Dokumenten oder Durschnitten von Korpora eine Abteilung für das Dokument identifiziert werden. 

%\textbf{Ein großer Korpus}\\
%Alle Dokumente ergeben einen Korpus. Dieser gibt eine Dokument-Topic-Verteilung aus, die eine generelle inhaltliche Verteilung der Anfragen darstellt. Neue Dokumente können auf dem Korpus geprüft werden, um eine Dokument-Topic-Verteilung zu inferieren. Diese ergibt eine Abteilung, dem das Dokument zugeordnet werden soll. 

%Ein Gesamtbild der thematischen Aufteilung aller Dokumente kann eine mögliche effektivere Neuverteilung der Kategorien schlussfolgern. Diese Methode stellt die simpleste Form der LDA Anwendung auf die ZVO Daten dar. Dafür kann man zwei verschiedene Ansätze verfolgen: 
%\begin{enumerate}
%	\item Mit 80\% der Daten wird das Modell trainiert. Die letzten 20 \% werden zum Testen verwendet. In diesem Fall wird aus dem unüberwachten LDA-Prinzip ein überwachtes Modell simuliert, da die händisch zugeteilten Abteilungen für die Dokumente bereit bekannt sind. Somit kann man prüfen, wie gut das Modell ist. Dies lässt sich in einer Quote darstellen, die die richtig zugeordneten Dokumente durch die Gesamtanzahl der Testdaten teilt. 
%	\item Mit 80\% der Daten wird das Modell trainiert. In dieser Methode werden jedoch die verwendeten 80\% der Daten verwendet, um zu testen, ob eine Veränderung der Korrektheit zu erkennen ist, wenn die  Trainingsdaten gleichzeitig die Testdaten sind. Dafür wird eine Matrix aufgetellt 
%\end{enumerate}

%\item \textbf{18 Korpora mit Durchschnitt}\\
%Jede Abteilung stellt einen Korpus da, deren Verteilung mit der des neuen Dokuments verglichen wird. In diesem Fall hat jede der 18 Abteilungen eine Dokument-Themen Verteilung, die die Abteilung inhaltlich von den anderen unterscheidet. Dazu wird ein neues Dokument in jedem einzelnen Korpus integriert, um eine Dokument-Themen Verteilung für das neue Dokument auf Basis der korpusabhängigen Wörter-Themen Verteilung zu inferieren. 

%\item \textbf{18 Korpora mit Mehrheitsentscheid}\\
%Jede Abteilung stellt einen Korpus dar, deren Dokumente alle einzeln mit der Verteilung des neuen Dokuments verglichen werden, das dann quantitativ einer Abteilung zugeordnet wird. Die Wörter-Themen Verteilung ist für alle dokumente eines Korpus gleich, während für jedes Dokument eine Dokument-Themen Verteilung errechnet wird. Wenn die Dokument-Themen Verteilung der bestehenden Dokumente und die des neuen Dokuments bereitstehen, können diese auf Ähnlichkeit überprüft werden. Da bereits bekannt ist, welcher Abteilung jedes Dokument angehört, stellen die Top X ähnlichsten Dokumente eine Verteilung der Abteilungen dar, denen das neue Dokument inhaltlich am ähnlichsten ist. 
\end{enumerate}



\section{Umsetzung Konzept}

\begin{enumerate}
\item
Alle Dokumente ergeben einen Korpus. Der Korpus generiert eine Topic-Verteilung für die Gesamtheit aller Dokumente. Dabei werden zuerst alle Anfragedaten in einen String zusammengefügt, der als Grundlage für das Wörterbuch und den Korpus dient. Um diesen in ein Dictionary, also eine numerierte Auflistung aller Wörter und dessen Anzahl, zu verwandelt, muss der String in eine Liste mit voneinander getrennten Items gesplittet werden. Hier wird ein Bag of Words Prinzip verfolgt, die Reihenfolge ist irrelevant für das Ergebnis des Modells. Aus der Liste wird dann das Dictionary erstellt. Durch den Aufruf des LDA Modells wird aus dem Bag of Words mithilfe des Dictionary eine vorgegebene Anzahl an Themen aus der Wortenge modelliert, basierend auf häufig zusammen auftretenden Wörtern. Dadurch ergibt sich neben einer Verteilung der Topics in einem Modell die Verteilung der Wörter, die ein Topic besonders beeinflussen. Ist das Modell generiert, können die Topics ausgelesen werden mit diesem Aufruf:\lstinline{pprint(lda.print_topics())}\\
\\

\begin{figure}[H]
\begin{lstlisting}[language=Python, basicstyle=\small]
INPUT:
data = ''

for x in range(0,106000):
    data += df.at[x,'subject-message']

list = data.split()

dictionary = corpora.Dictionary([list])
temp = dictionary[0]
id2word = dictionary.id2token

corpus = [dictionary.doc2bow(text) for text in [list]]

lda = LdaModel(corpus, num_topics=18, id2word = id2word)

pprint(lda.print_topics())

\end{lstlisting}
\caption{Der Input zum Aufbau eines LDA Modells}
\end{figure}

\begin{figure}[H]
\begin{lstlisting}[language=Python]
OUTPUT:
[(0,
  '0.012*"ostholstein" + 0.011*"nachricht" + 0.010*"sierksdorf" + '
  '0.009*"zweckverband" + 0.008*"betreff" + 0.008*"danken" + 0.007*"email" + '
  '0.007*"hra" + 0.007*"datum" + 0.007*"hyperlink"'),
 (1,
  '0.014*"ostholstein" + 0.011*"nachricht" + 0.010*"zweckverband" + '
  '0.009*"sierksdorf" + 0.008*"sitzen" + 0.008*"danken" + 0.007*"wagrienring" '
  '+ 0.007*"lübeck" + 0.007*"betreff" + 0.006*"the"'),
 (2,
  '0.014*"sierksdorf" + 0.013*"zweckverband" + 0.011*"nachricht" + '
  '0.010*"ostholstein" + 0.008*"danken" + 0.008*"betreff" + 0.007*"sitzen" + '
  '0.006*"the" + 0.006*"homepage" + 0.006*"frau"'),
 (3,
  '0.014*"sierksdorf" + 0.011*"ostholstein" + 0.011*"nachricht" + '
  '0.010*"zweckverband" + 0.010*"betreff" + 0.008*"the" + 0.007*"frau" + '
  '0.006*"danken" + 0.006*"denken" + 0.006*"lübeck"'),
 (4,
  '0.012*"zweckverband" + 0.011*"frau" + 0.011*"sierksdorf" + '
  '0.011*"ostholstein" + 0.009*"nachricht" + 0.008*"betreff" + 0.007*"the" + '
  '0.007*"danken" + 0.006*"lübeck" + 0.006*"öffentlich"'),
 (5,
  '0.015*"zweckverband" + 0.011*"sierksdorf" + 0.011*"ostholstein" + '
  '0.009*"the" + 0.008*"betreff" + 0.007*"lübeck" + 0.007*"danken" + '
  '0.007*"nachricht" + 0.007*"hyperlink" + 0.006*"sitzen"'),
 (6,
  '0.013*"nachricht" + 0.012*"ostholstein" + 0.012*"zweckverband" + '
  '0.010*"betreff" + 0.009*"sierksdorf" + 0.007*"danken" + 0.006*"the" + '
  '0.006*"lübeck" + 0.006*"frau" + 0.005*"hyperlink"'),
 (7,
  '0.012*"ostholstein" + 0.012*"sierksdorf" + 0.011*"zweckverband" + '
  '0.010*"nachricht" + 0.008*"danken" + 0.008*"the" + 0.007*"hra" + '
  '0.007*"email" + 0.007*"betreff" + 0.006*"lübeck"'),
 (8,
  '0.012*"ostholstein" + 0.009*"sierksdorf" + 0.008*"nachricht" + '
  '0.008*"danken" + 0.008*"betreff" + 0.008*"frau" + 0.008*"zweckverband" + '
  '0.008*"hyperlink" + 0.007*"email" + 0.007*"homepage"'),
 (9,
  '0.013*"sierksdorf" + 0.012*"zweckverband" + 0.011*"ostholstein" + '
  '0.009*"nachricht" + 0.009*"danken" + 0.008*"frau" + 0.007*"the" + '
  '0.007*"lübeck" + 0.006*"hyperlink" + 0.006*"sitzen"'),
 (10,
  '0.011*"zweckverband" + 0.011*"nachricht" + 0.010*"sierksdorf" + '
  '0.010*"betreff" + 0.009*"ostholstein" + 0.009*"lübeck" + 0.007*"the" + '
  '0.007*"sitzen" + 0.007*"danken" + 0.007*"hyperlink"'),
 (11,
  '0.013*"sierksdorf" + 0.009*"ostholstein" + 0.009*"zweckverband" + '
  '0.008*"betreff" + 0.008*"nachricht" + 0.007*"danken" + 0.007*"hyperlink" + '
  '0.007*"lübeck" + 0.006*"email" + 0.006*"datum"'),
  
  \end{lstlisting}
  \end{figure}
  \newpage
  \begin{figure}[h]
  \begin{lstlisting}
 (12,
  '0.013*"ostholstein" + 0.012*"sierksdorf" + 0.009*"nachricht" + '
  '0.009*"zweckverband" + 0.008*"betreff" + 0.008*"danken" + 0.007*"email" + '
  '0.006*"frau" + 0.006*"lübeck" + 0.006*"wagrienring"'),
 (13,
  '0.013*"sierksdorf" + 0.010*"the" + 0.010*"zweckverband" + '
  '0.009*"ostholstein" + 0.009*"nachricht" + 0.008*"danken" + 0.007*"lübeck" + '
  '0.007*"betreff" + 0.007*"frau" + 0.007*"öffentlich"'),
 (14,
  '0.012*"zweckverband" + 0.011*"danken" + 0.011*"sierksdorf" + '
  '0.010*"ostholstein" + 0.010*"betreff" + 0.008*"nachricht" + '
  '0.007*"hyperlink" + 0.007*"gmbh" + 0.006*"lübeck" + 0.006*"kundennummer"'),
 (15,
  '0.014*"ostholstein" + 0.011*"sierksdorf" + 0.008*"zweckverband" + '
  '0.008*"danken" + 0.008*"frau" + 0.007*"betreff" + 0.007*"the" + '
  '0.007*"nachricht" + 0.006*"amtsgericht" + 0.006*"sitzen"'),
 (16,
  '0.014*"zweckverband" + 0.013*"ostholstein" + 0.011*"nachricht" + '
  '0.009*"danken" + 0.008*"sierksdorf" + 0.007*"hyperlink" + 0.007*"lübeck" + '
  '0.006*"sitzen" + 0.006*"the" + 0.006*"betreff"'),
 (17,
  '0.012*"ostholstein" + 0.011*"danken" + 0.011*"sierksdorf" + 0.009*"betreff" '
  '+ 0.009*"zweckverband" + 0.008*"nachricht" + 0.007*"the" + 0.007*"frau" + '
  '0.006*"hyperlink" + 0.006*"sitzen"')]
\end{lstlisting}
\caption{Die Topics und Schlüsselwörter}
\end{figure}

Das LDA hat erfolgreich auf allen verfügbaren Daten ein Modell gebaut, das die Daten in Topics $[0,17]$ eingeteilt hat. Jedes Topic hat dominante Wörter, die zusammen mit ihren jeweiligen Auftretenswahrscheinlichkeiten aufgeslistet sind. Die Wahrscheinlichkeiten für das Auftreten von Wörtern liegt im Intervall $[1.5\%,0\%]$. Es ist auffällig, dass sich die Themen viele der wahrscheinlichsten Wörter teilen. \\


\item Die Qualität der Daten und des Klassifikators kann alternativ über die Zuordnung der Topics zu den Abeteilungen untersucht werden. Der wohl naivste Ansatz ist, Topic $x$ auf Abteilung $x$ für alle $x$ in $[0,17]$ abzubilden. Die folgende Graphik zeigt diesen Ansatz durch die Gegenüberstellung der jeweils zugeordneten Dokumente. Dabei wird nur die Anzahl der zugeordneten Dokumente überprüft, nicht, die Dokumente, die in beiden vorliegen. Überschneidungen werden noch nicht betrachtet.

\begin{center}
\begin{tikzpicture}
 
\begin{axis} [tick label style={/pgf/number format/fixed}, scaled ticks=false,ybar,xmin=-1,xmax=18,ymin=0,ymax=30000, bar width=8pt, height=8cm,width=15cm, ylabel={Anzahl Elemente}, xlabel={Topics bzw. Abteilungen}]
\addplot coordinates {
    (0,8605)
    (1,3959)
    (2,5974)
    (3,1714)
    (4,5823)
    (5,2291)
    (6,3558)
    (7,12721)
    (8,4207)
    (9,1100)
    (10,10021)
    (11,3243)
    (12,14556)
    (13,2697)
    (14,29456)
    (15,1162)
    (16,575)
    (17,1400)
};
\addplot coordinates {
(0,9770)
(1,21061)
(2,14245)
(3,235)
(4,1307)
(5,251)
(6,4009)
(7,3610)
(8,9410)
(9,23533)
(10,20676)
(11,43)
(12,2866)
(13,9616)
(14,8144)
(15,12126)
(16,4748)
(17,2700)
};

\legend{LDA, Label};
\end{axis}
\caption{Vergleich der Anzahl an Dokumenten pro Topic bzw. Abteilung}
\end{tikzpicture}
\end{center}

Das Säulendiagramm zeigt, dass die Topics nicht optimal auf die Abteilungen abgebildet sind. Die Anzahl der enthaltenen Dokumente sollte ähnlicher sein. Betrachtet man zum Beispiel Topic 14, kann diese nicht Abteilung 14 darstellen, da es sich um einen Unterschied von 21312 Dokumenten, also 72$\%$, handelt. Die genaue Anzahl der Zuteilungen zu Topics bzw Abteilungen ist in folgenden Tabellen gelistet: 

\begin{figure}[H]
\begin{center}
\begin{tabular}{cc}
\hline 
\hline
Topic&Counts LDA\\
\hline
0&8605\\
1&3959\\
2&5974\\
3&1714\\
4&5823\\
5&2291\\
6&23558\\
7&12721\\
8&4207\\
9&1100\\
10&10021\\
11&3243\\
12&14556\\
13&2697\\
14&29456\\
15&1162\\
16&575\\
17&1400\\
\hline
\hline
\end{tabular}
\caption{Anzahl Dokumente in den Topics}
\end{center}
\end{figure}

%\quad

\begin{figure}[H]
\begin{center}
\begin{tabular}{cc}
\hline
\hline
Abteilung & Counts Label\\
\hline
0&9770\\
1&21061\\
2&14245\\
3&235\\
4&1307\\
5&251\\
6&4009\\
7&3610\\
8&9410\\
9&23533\\
10&20676\\
11&43\\
12&2866\\
13&9616\\
14&8144\\
15&12126\\
16&4748\\
17&2700\\
\hline
\hline
\end{tabular}
\caption{Anzahl Dokumente in den Abteilungen}
\end{center}
\end{figure}


Die gleichen Daten sind hier in absteigender Reihenfolge dargestellt. Dabei wurde der Anteil an den Gesamtdaten hinzugefügt, um einen Vergleichswert zu haben:


\begin{figure}[H]
\begin{center}
\begin{tabular}{ccc}
\hline 
\hline
Topics & Counts LDA & Anteil$[\%]$\\
\hline
14&29456&22.13\\
6&23558&17.7\\
12&14556&10.94\\
7&12721&9.56\\
10&10021&7.53\\
0&8605&6.47\\
2&5974&4.49\\
4&5823&4,38\\
8&4207&3.16\\
1&3959&2.98\\
11&3243&2.44\\
13&2697&2.03\\
5&2291&1.72\\
3&1714&1.29\\
17&1400&1.05\\
15&1162&0.87\\
9&1100&0.83\\
16&575&0.43\\
\hline
\hline
\end{tabular}
\caption{Anzahl Dokumente in Topics mit relativem Anteil}
\end{center}
\end{figure}

%\quad

\begin{figure}[H]
\begin{center}
\begin{tabular}{ccc}
\hline
\hline
Abteilung & Counts Label & Anteil$[\%]$\\
\hline
9&23533&15.86\\
1&21061&14.20\\
10&20676&13.94\\
2&14245&9.6\\
15&12126&8.17\\
0&9770&6.59\\
13&9616&6.48\\
8&9410&6.34\\
14&8144&5.49\\
16&4748&3.20\\
6&4009&2.7\\
7&3610&2.43\\
12&2866&1.93\\
17&2700&1.82\\
4&1307&0.88\\
5&251& 0.17\\
3&235&0.16\\
11&43&0.03\\
\hline
\hline
\end{tabular}
\caption{Anzahl Dokumente in Abteilungen mit relativem Anteil}
\end{center}
\end{figure}


Nun wird durch die quantitativere Darstellung deutlich, dass sich nicht alle Topics eindeutig einer Abteilung zuordnen lassen. Das Ergebnis zeigt, dass die Topic-Verteilung bei LDA das 14. Thema stärker erkennt, als die ZVO die Abteilung mit den meisten zugeordneten Dokumenten. Diese unterschieden sich durch $5923$ Dokumente. Die Betrachtung der unterschiedlichen Anteile dient somit nicht gut als Zuordnungsmethode, kann jedoch Aufschluss darüber geben, dass die Daten von der ZVO gleichmäßiger zugeteilt wurden, als sie von dem Topic-Modelling erkannt werden. Dies bedeutet, dass die Wörter in den Dokumenten in ihrer Bedeutung und syntaktischen Umgebung zu ähnlich sind, als dass sie sich optimal in die von der ZVO vorgegebenen Abteilungen einteilen lassen. Dies könnte auch an bestimmten Wörtern liegen, die in vielen Abteilungen vorkommen, wie zu, Beispiel 'Ostholstein' oder 'Zweckverband'. Die Wörter können dafür sorgen, dass Dokumente in die gleiche Topic sortiert werden, die sich eigentlich durch andere Wörter semantisch stärker voneinander unterscheiden und manuell anderen Abteilungen untergeordnet wurden.\\ 



Den Topics ist also keine Verbindung zu den gegebenen Abteilungen direkt zuzuordnen. Diese muss über andere Weise herausgefunden werden, damit die Zuordnung der Dokumente evaluiert werden kann. Die Zuordnung der Topics zu Abteilungen kann approximiert werden, indem gezählt wird, in wie vielen Dokumenten sich die Topics und Abteilungen jeweils überschneiden. Genau wie bei der Gruppeneinteilung wird für diese Überschneidung jedes Dokument erneut auf dem Korpus geprüft, um dessen Topic-Wort-Verteilung zu erstellen. Die Dokumente werden anhand ihrer einzigartigen Zeilen-ID adressiert. Die IDs werden der Topic zugeordnet, die die höchste Wahrscheinlichkeit in der Verteilung des Dokuments erreicht. Die Zuordnung findet in einer großen Liste statt, die über 18 Unterlisten verfügt, in die die IDs jeweils hinzugefügt werden. Die Überschneidungen werden in einer Matrix dargestellt, die für jedes Topic des Modells die Überschneidungen zu jeder Abteilung auflistet. Somit kann für jede Zeile die passende Zuordnung gefunden werden. Dabei stellt eine Zeile jeweils die Dokumente dar, die aufsummiert die Zahl der Tabelle X (vorherige) ergeben: 

%comp bei 10000: [[536, 91, 67, 1, 12, 1, 12, 6, 23, 145, 69, 0, 22, 12, 38, 87, 24, 6], [15, 22, 11, 0, 2, 0, 0, 1, 7, 32, 77, 0, 6, 3, 5, 16, 3, 1], [6, 50, 81, 1, 3, 1, 31, 6, 28, 61, 30, 0, 18, 15, 22, 33, 4, 16], [14, 69, 58, 3, 3, 0, 19, 11, 37, 104, 63, 0, 18, 20, 21, 51, 33, 5], [7, 81, 104, 1, 3, 0, 7, 9, 56, 112, 97, 1, 11, 28, 36, 42, 15, 12], [0, 4, 3, 1, 0, 0, 1, 0, 2, 5, 3, 0, 1, 1, 0, 1, 0, 0], [2, 7, 24, 0, 0, 0, 2, 0, 7, 25, 23, 0, 2, 5, 2, 13, 8, 1], [19, 331, 90, 3, 16, 3, 49, 22, 111, 206, 200, 2, 29, 49, 61, 127, 36, 44], [3, 20, 33, 0, 1, 0, 2, 1, 11, 41, 78, 0, 3, 5, 7, 15, 8, 1], [4, 21, 17, 0, 2, 0, 1, 2, 26, 18, 33, 0, 3, 9, 5, 4, 2, 0], [13, 137, 115, 0, 6, 0, 25, 16, 53, 183, 136, 3, 6, 34, 75, 57, 39, 9], [27, 429, 221, 8, 34, 4, 129, 34, 291, 312, 271, 2, 48, 137, 75, 149, 50, 106], [0, 12, 8, 0, 1, 0, 3, 2, 5, 24, 13, 0, 2, 1, 3, 10, 5, 1], [12, 89, 73, 0, 9, 1, 9, 18, 25, 106, 124, 0, 4, 14, 29, 69, 52, 5], [10, 174, 110, 4, 11, 1, 15, 5, 42, 226, 137, 1, 19, 38, 39, 181, 21, 27], [8, 67, 70, 1, 4, 0, 11, 4, 21, 85, 115, 0, 8, 16, 20, 47, 21, 14], [19, 33, 44, 0, 5, 0, 4, 4, 18, 56, 63, 0, 6, 9, 16, 18, 7, 3], [39, 47, 52, 0, 5, 1, 15, 10, 203, 74, 71, 1, 12, 13, 16, 33, 17, 6]]

\begin{figure}[h]
\begin{center}
\tiny
\renewcommand{\arraystretch}{2}
\begin{tabular}{ccccccccccccccccccc}
\hline
\hline 
&0&1&2&3&4&5&6&7&8&9&10&11&12&13&14&15&16&17\\
\hline
\hline
0&555& 977& \textbf{2150}& 21& 66& 19& 211& 251& 910& 960& 902& 2& 210& 596& 302& 758& 318& 250\\
\hline 
1&512& 580& \textbf{650}& 3& 36& 7& 139& 101& 370& 453& 570& 0& 66& 378& 91& 196& 123& 53\\
\hline 
2&508& 755& 263& 9& 67& 21& 254& 145& 282& \textbf{1498}& 1207& 1& 119& 300& 479& 527& 177& 130\\
\hline 
3&108& 179& 98& 2& 15& 3& 33& 57& 236& \textbf{432}& 218& 0& 34& 55& 153& 198& 74& 27\\
\hline 
4&493& 867& 442& 5& 63& 8& 134& 180& 451& \textbf{1111}& 1033& 8& 73& 348& 446& 489& 191& 89\\
\hline 
5&91& 467& 98& 5& 25& 2& 60& 75& 162& \textbf{480}& 309& 1& 55& 127& 169& 351& 113& 47\\
\hline 
6&1263& 3993& 1738& 28& 127& 48& 811& 587& 1384& 4139& \textbf{4730}& 6& 573& 2096& 1620& 2080& 680& 495\\
\hline 
7&\textbf{3376}& 1298& 870& 17& 524& 16& 221& 338& 662& 1904& 1740& 9& 185& 381& 664& 891& 537& 133\\
\hline 
8&310& 585& 216& 10& 27& 5& 67& 114& 203& 771 & \textbf{1013}& 0& 81& 323& 264& 440& 205& 73\\
\hline 
9&58& 126& 47& 0& 3& 1& 17& 16& 131& 174& \textbf{400}& 0& 15& 55& 63& 69& 59& 17\\
\hline 
10&286& \textbf{1610}& 1733& 12& 51& 27& 284& 299& 858& 1567& 1331& 4& 239& 893& 674& 673& 304& 248\\
\hline 
11&194& 409& 377& 8& 27& 3& 88& 122& 180& \textbf{678}& 470& 0& 80& 173& 167& 395& 142& 118\\
\hline 
12&486& 2355& 2495& 16& 80& 24& 564& 327& 1005& \textbf{2672}& 1513& 3& 311& 1254& 886& 1377& 579& 327\\
\hline 
13&151& 393& 198& 5& 23& 4& 64& 83& 160& 414& \textbf{639}& 1& 68& 192& 123& 275& 151& 68\\
\hline 
14&1008& 6084& 2681& 86& 153& 61& 990& 820& 2223& \textbf{5730}& 3832& 8& 701& 2327& 1860& 3093& 993& 552\\
\hline 
15&270& 95& 36& 3& 10& 1& 29& 32& 46& 228& \textbf{284}& 0& 19& 31& 61& 81& 23& 21\\
\hline 
16&81& 82& 25& 1& 3& 0& 18& 38& 27& \textbf{105}& 90& 0& 14& 29& 37& 76& 24& 15\\
\hline
17&20& 206& 128& 4& 7& 1& 25& 25& 120& \textbf{217}& 395& 0& 23& 58& 85& 157& 55& 37\\
\hline
\hline
\end{tabular}
\caption{Anzahl der Überschneidungen zwischen LDA Topics (vert.) und ZVO Labels (horiz.)}
\end{center}
\end{figure}\\


%[[555, 977, 2150, 21, 66, 19, 211, 251, 910, 960, 902, 2, 210, 596, 302, 758, 318, 250], [512, 580, 650, 3, 36, 7, 139, 101, 370, 453, 570, 0, 66, 378, 91, 196, 123, 53], [508, 755, 263, 9, 67, 21, 254, 145, 282, 1498, 1207, 1, 119, 300, 479, 527, 177, 130], [108, 179, 98, 2, 15, 3, 33, 57, 236, 432, 218, 0, 34, 55, 153, 198, 74, 27], [493, 867, 442, 5, 63, 8, 134, 180, 451, 1111, 1033, 8, 73, 348, 446, 489, 191, 89], [91, 467, 98, 5, 25, 2, 60, 75, 162, 480, 309, 1, 55, 127, 169, 351, 113, 47], [1263, 3993, 1738, 28, 127, 48, 811, 587, 1384, 4139, 4730, 6, 573, 2096, 1620, 2080, 680, 495], [3376, 1298, 870, 17, 524, 16, 221, 338, 662, 1904, 1740, 9, 185, 381, 664, 891, 537, 133], [310, 585, 216, 10, 27, 5, 67, 114, 203, 771, 1013, 0, 81, 323, 264, 440, 205, 73], [58, 126, 47, 0, 3, 1, 17, 16, 131, 174, 400, 0, 15, 55, 63, 69, 59, 17], [286, 1610, 1733, 12, 51, 27, 284, 299, 858, 1567, 1331, 4, 239, 893, 674, 673, 304, 248], [194, 409, 377, 8, 27, 3, 88, 122, 180, 678, 470, 0, 80, 173, 167, 395, 142, 118], [486, 2355, 2495, 16, 80, 24, 564, 327, 1005, 2672, 1513, 3, 311, 1254, 886, 1377, 579, 327], [151, 393, 198, 5, 23, 4, 64, 83, 160, 414, 639, 1, 68, 192, 123, 275, 151, 68], [1008, 6084, 2681, 86, 153, 61, 990, 820, 2223, 5730, 3832, 8, 701, 2327, 1860, 3093, 993, 552], [270, 95, 36, 3, 10, 1, 29, 32, 46, 228, 284, 0, 19, 31, 61, 81, 23, 21], [81, 82, 25, 1, 3, 0, 18, 38, 27, 105, 90, 0, 14, 29, 37, 76, 24, 15], [20, 206, 128, 4, 7, 1, 25, 25, 120, 217, 395, 0, 23, 58, 85, 157, 55, 37]] MIT 133044 Dokumenten!!!!

Diese Matrix bildet die Überschneidungen aller 133044 Dokumenten dar. Dabei sind die 18 LDA Topics auf der vertikalen Achse und die 18 LABEL Abteilungen auf der horizontalen Achse aufgetragen. Die Ausführung auf diesen Daten hat bereits mehrere Tage gedauert. Bezüglich der Zuordnung ist der naivste Ansatz, jeder Zeile (also LDA Topic) die Spalte (also Label Abteilung) mit der maximalen Überschneidung zuzuordnen. Dabei ist der Ziel die Anzahl der Gesamtüberschneidungen zu maximieren. Wendet man diesen Algorithmus an, sieht der Output wie folgt aus: \\

\begin{figure}[H]
\begin{lstlisting}[language=Python]
maxmatrix = [2,2,9,9,9,9,10,0,10,10,1,9,9,10,9,10,9,9]
\end{lstlisting}
\caption{Liste der zugeordneten Topics und Abteilungen}
\end{figure}

Wie zu erkennen ist, sind nur die Zahlen $ 0,1,2,9,10 $ in der Liste vertreten. Eine optimale Zuordnung wäre jedoch erst erreicht, wenn alle Zahlen von Eins bis 18 ohne Duplikate in der Liste in einer beliebigen Reihenfolge vorkommen. Dies liegt daran, dass eine Topic nur genau eine Abteilung darstellen darf. In der Liste wird jedoch deutlich, dass viele Themen des LDA Modells mit dem neunten Thema der ZVO kompatibel wären. Dabei werden $9/18$ Topics Abteilung $9$, $1/18$ Topic Abteilung $1$, $5/18$ Topics Abteilung $10$ und $2/18$ Topics Abteilung $2$ zugeordnet. Hier fällt auf, dass diesen Abteilungen die meisten Dokumente von der ZVO händisch zugeordnet wurden. Somit können sie sich auch mit vielen Dokumenten aus den LDA Topics überschneiden. \textbf{Dies lässt zusätzlich vermuten, dass zum Beispiel die Abteilung 9, die laut Tabelle X die meisten Dokumente enthält, ein zu breites inhaltliches Spektrum abdeckt und in noch weitere Unterthemen unterteilt werden könnte. Dies wird dadurch begründet, dass das Topic-Modell viele der Dokumente aus Abteilung 9 in verschiedene Topics eingeteilt hat.}\\


Die Zuteilung anhand der maximalen absoluten Überschneidungen ist nicht injektiv. Für die Verarbeitung der Daten ist die injektive Zuordnung jedoch fundamental wichtig, aber schwierig zu erreichen. Für die Beurteilung der Matrix führen wir einen weiteren Parameter ein, die durchschnittliche Überschneidung. Dadurch kann die Kompatibilität der Abteilungen im Bezug auf das Matching mit Topics besser analysiert werden. Vorallem bei Abteilungen, die eine sehr unausgeglichene Überschneidungsmengen haben, da sie deutlich mehr Dokumente als andere enthalten, bietet der Durchschnitt eine alternative Sichtweise. Für die durchschnittliche Überschneidung wird zuerst die Summe aller Dokumente einer Abteilung errechnet, indem alle Werte einer Spalte addiert werden. Daraufhin entsteht eine neue Matrix, in der jeder Wert jeweils durch die Summe seiner Spalte dividiert wird. Dabei wird die Anzahl an Dokumenten in jeder Abteilung irrelevant für das Endergebnis. Somit beschreibt jeder Wert, wie groß der Anteil der aus dieser Topic überschneidenden Dokumente bezogen auf die gesamte Abteilung ist in $\%$: \\

\begin{figure}[H]
\begin{center}
\tiny
\renewcommand{\arraystretch}{2}
\begin{tabular}{ccccccccccccccccccc}
\hline
\hline
&0&1&2&3&4&5&6&7&8&9&10&11&12&13&14&15&16&17\\
\hline
0&5.68&4.64&15.09&8.94&5.05&7.57&5.26&6.95&9.67&4.08&4.36&4.65&7.33&6.2&3.71&6.25&6.7&9.26\\\hline
1&5.24&2.75&4.56&1.28&2.75&2.79&3.47&2.8&3.93&1.92&2.76&0&2.3&3.93&1.12&1.62&2.59&1.96\\\hline
2&5.2&3.58&1.85&3.83&5.13&8.37&6.34&4.02&3&6.37&5.84&2.33&4.15&3.12&5.88&4.35&3.73&4.81\\\hline
3&1.11&0.85&0.69&0.85&1.15&1.2&0.82&1.58&2.51&1.84&1.05&0&1.19&0.57&1.88&1.63&1.56&1\\\hline
4&5.05&4.12&3.1&2.13&4.82&3.19&3.34&4.99&4.79&4.72&5&18.6&2.55&3.62&5.48&4.03&4.02&3.3\\\hline
5&0.93&2.22&0.69&2.13&1.91&0.8&1.5&2.08&1.72&2.04&1.49&2.33&1.92&1.32&2.08&2.89&2.38&1.74\\\hline
6&12.93&18.96&12.2&11.91&9.72&19.12&20.23&16.26&14.71&17.59&22.88&13.95&19.99&21.8&19.89&17.15&14.32&18.33\\\hline
7&34.55&6.16&6.11&7.23&40.09&6.37&5.51&9.36&7.04&8.09&8.42&20.93&6.45&3.96&8.15&7.35&11.31&4.93\\\hline
8&3.17&2.78&1.52&4.26&2.07&1.99&1.67&3.16&2.16&3.28&4.9&0&2.83&3.36&3.24&3.63&4.32&2.7\\\hline
9&0.59&0.6&0.33&0&0.23&0.4&0.42&0.44&1.39&0.74&1.93&0&0.52&0.57&0.77&0.57&1.24&0.63\\\hline
10&2.93&7.64&12.17&5.11&3.9&10.76&7.08&8.28&9.12&6.66&6.44&9.3&8.34&9.29&8.28&5.55&6.4&9.19\\\hline
11&1.99&1.94&2.65&3.4&2.07&1.2&2.2&3.38&1.91&2.88&2.27&0&2.79&1.8&2.05&3.26&2.99&4.37\\\hline
12&4.97&11.18&17.51&6.81&6.12&9.56&14.07&9.06&10.68&11.35&7.32&6.98&10.85&13.04&10.88&11.36&12.19&12.11\\\hline
13&1.55&1.87&1.39&2.13&1.76&1.59&1.6&2.3&1.7&1.76&3.09&2.33&2.37&2&1.51&2.27&3.18&2.52\\\hline
14&10.32&28.89&18.82&36.6&11.71&24.3&24.69&22.71&23.62&24.35&18.53&18.6&24.46&24.2&22.84&25.51&20.91&20.44\\\hline
15&2.76&0.45&0.25&1.28&0.77&0.4&0.72&0.89&0.49&0.97&1.37&0&0.66&0.32&0.75&0.67&0.48&0.78\\\hline
16&0.83&0.39&0.18&0.43&0.23&0&0.45&1.05&0.29&0.45&0.44&0&0.49&0.3&0.45&0.63&0.51&0.56\\\hline
17&0.2&0.98&0.9&1.7&0.54&0.4&0.62&0.69&1.28&0.92&1.91&0&0.8&0.6&1.04&1.29&1.16&1.37\\\hline\hline
$\sum$&100&100&100&100&100&100&100&100&100&100&100&100&100&100&100&100&100&100\\\hline \hline
\end{tabular}
\caption{Durchschnittliche Überschneidungen aller Topics mit Labels}
\end{center}
\end{figure}

Wählt man aus jeder Zeile das maximale Element aus und notiert die assoziierten Abteilungen in einer Liste, dann ergibt sich das folgende Ergebnis. Der Unterschied bei der Darstellung ist nun, dass die Zahlen pro Zeile nicht mehr als absolutes Maß betrachtet werden, sonder als relative Häufigkeit in der jeweiligen Abteilung. Somit macht es keinen Unterschied, dass die Abteilungen verschieden viele Dokumente von der ZVO zugeordnet bekommen haben. \\

\begin{figure}[h]
\begin{center}
\begin{lstlisting}[language=Python]
average_row = [2,0,5,8,11,15,10,4,10,10,2,17,2,16,3,0,7,10]
\end{lstlisting}
\caption{Zuordnung zwischen LDA und Label} 
\end{center}
\end{figure}\\

Man erkennt bereits eine Verbesserung zu Figure X (absolute Max Zuteilung), da in der Zuteilung mehr Abteilungen zugeordnet wurden. Insgesamt sind $12/18$ in der Liste vertreten. Wie bei der ersten Liste, bei der  $ 0,1,2,9,10 $ mehrfach zugeweisen wurden, sind hier $2$ und $10$ drei mal zugewiesen und $0$ zwei mal. Somit unterstreicht das Ergebnis die Problematik der Zuteilung der ZVO in Bezug auf diese Abteilungen. \\

\end{enumerate}

	
\chapter{Analyse}%%%%%%%%%%%%%%%%%%%%%%%%%%%%%%%%%%%%%%%%%%%%%%%%%%%%%%%%%ANALYSE
Die Implementierung des LDA Modells wurde im vorausgegangenen Teil umegsetzt. Die Ausgaben werden nun analysiert....

\begin{enumerate}
\item \textbf{Gruppen und Wörter finden}\\
Das LDA Modell hat 18 Gruppen aus den Daten generiert und diese mit einer Verteilung von Schlüsselwörtern beschrieben. Als Anwendungsziel in Bezug auf die ZVO sollen die Topics des Modells in Zukunft die Abteilungen darstellen. Betrachtet man dafür die Wörter der Topics, um die Topics semantisch den Abteilungen zuzuordnen, wird dies nicht gelingen. Das liegt daran, dass sich die dominaten Wörter in den Topics zu stark überschneiden. Zum Beispiel gehört \lstlinline{"ostholstein"} in allen Topics zu den Top 5 Wörtern. In der folgenden Darstellung sind die Top 5 Wörter aufgelistet und mit den 18 Topics auf Vorkommen geprüft: 

\begin{figure}[H]
\begin{center}
%\tiny
%\renewcommand{\arraystretch}{2}
%\begin{tabular}{ccccccccccccccccccc}
%Wort&L0&L1&L2&L3&L4&L5&L6&L7&L8&L9&L10&L11&L12&L13&L14&L15&L16&L17&L18\\
\begin{tikzpicture}
  \begin{axis}[title  = Topicabdeckung der Top 5 Schlüsselwörter,
    xbar,
    y axis line style = { opacity = 0 },
    axis x line       = none,
    tickwidth         = 0pt,
    ytick             = data,
    enlarge y limits  = 0.05,
    enlarge x limits  = 0.02,
    width=0.7\textwidth,,
    bar width=5.5mm,    symbolic y coords = {oosthosltein,sierksdorf,zweckverband,nachricht,betreff,danken,frau,the,sitzen}, nodes near coords
  ]
  \addplot coordinates { (18,ostholstein)(17,zweckverband)(18,sierksdorf)(14,nachricht)(10,betreff)(8,danken)(2,frau)(2,the)(1,sitzen)};
  \end{axis}
\end{tikzpicture}
\caption{Topicabdeckung der Top 5 Wörter aller Topics}
\end{center}
\end{figure}

Das bedeutet, dass sich die Themen semantisch nicht stark genug von einander abgrenzen lassen, da die Wörter zu ähnlich sind. Dies lässt auf die Folgerung schließen, dass die Daten bei der Datenreinigung noch stärker um die  häufigen Wörter reduziert werden sollten, um eine bessere Qualität zu erreichen. Dabei ist es jedoch wichtig, nicht die entscheidenen Topic-relevanten Wörter auszuschließen.
\\
\item \textbf{Zuordnung Abteilung zu Topic}\\
Die durchschnittliche Überschneidung zwischen Abteilung und Topic spiegelt wider, wie stark die Abteilung mit einem oder mehreren Topics übereinstimmt. Ein hoher Durschnitt kann entweder eine sehr hohe Überschneidung der Abteilung mit einer Topic oder relativ hohe Überschneidungen mit mehreren Topics bedeuten. Konkret für die Abteilungen bedeutet das: 
\end{enumerate}


\chapter{Zusammenfassung und Ausblick}%%%%%%%%%%%%%%%%%%%%%%%%%%%%%%%%%%%%%%%%%%%%%%%%ZUSAMMENFASSUNG
% In a German thesis write: \subsection{Zusammenfassung und Ausblick}


% !!!!!!!!!!!!!!!!!!!!!!!!!!!!!!!!!!
% !!! Your action is needed here !!!
% !!!!!!!!!!!!!!!!!!!!!!!!!!!!!!!!!!
%
% Replace the following with your conclusion



...



% Normally, the bibliography comes next at this point. Do *not* (try
% to) include further indices and tables like an index or
% a list of figures or a list of tables or such things. Nobody
% actually uses them and they just use up space. 
%
% You *can* however include a glossary, if this seems appropriate. It
% goes here as an unnumbered chapter. Most thesis will *not* need a
% glossary: a well-written text (re)explains strange words and
% concepts as necessary. However, there are situations where a
% glossary may be helpful.














%%%
% 
% Bibliographies
%
%%%
%
% The uzl-thesis class will load biblatex for the bibliography
% management. This is a powerful package, see its documentation for
% details. The styles will be setup correctly and automatically by
% choosing one of the two style keys as described earlier.
%
% In order for the bibliography to work, run latex in the following
% order (which is the standard order):
% 
% > lualatex thesis-example
% > bibtex thesis-example
% > lualatex thesis-example
% 
% Add BibTeX files using \addbibresource or use the {bibtex entries}
% environment (see below).
%
%%%
%
% Although everyting is normally setup automatically, you can change
% the options passed to biblatex using the key 'biblatex';
% for instance,
%
%   \UzLThesisSetup{biblatex={firstinits=false}}
%
% will switch off shortened first names. Normally, you will not need
% this key in your preamble. 
% 
% Note that the bibtex program is used as the 'backend' of biblatex
% by default (rather than biber, which is the preferred program of
% biblatex). This means that you can (and must) run *bibtex* after you
% have run lualatex on your thesis. If you wish to use biber instead
% of bibtex, say 'biblatex={backend=biber}'. 
% 
%%%
%
% The following environment is optional. It allows you to keep the
% bibtex entries for your thesis right here in the thesis file. What
% happens is that each time this tex file is processed, the contents
% of the following environment gets written to the file
% \jobname-bibtex-entries.bib (this file gets overwritten each
% time). Independently, \addbibresource{\jobname-bibtex-entries.bib}
% is always called if the file \jobname-bibtex-entries.bib
% exists. 
%
% In result, you can edit and keep the bibliography's bibtex entries
% right here. If you change something here, run latex, then bibtex,
% then latex once more.
%
% If you would like to manage the bibtex entries in a separate file,
% remove the below environment, delete the \jobname-bibtex-entries.bib
% file and instead write
%
% \addbibresource{filename-of-your-bibtex-file.bib}
%
% in the preamble.
%
%%%


% !!!!!!!!!!!!!!!!!!!!!!!!!!!!!!!!!!
% !!! Your action is needed here !!!
% !!!!!!!!!!!!!!!!!!!!!!!!!!!!!!!!!!
%
% Replace following example entries with the ones of your thesis.

\begin{bibtex-entries}

@Book{Knuth1986,
  author =       {Donald Erwin Knuth},
  title =        {The \TeX book},
  publisher =    {Addison-Wesley},
  year =         {1986},
}

@Book{Lamport1994,
  author =       {Leslie Lamport},
  title =        {\LaTeX: A Document Preparation System},
  publisher =    {Addison-Wesley},
  edition =      {Second edition},
  year =         {1994},
}

@TechReport{Kernighan1974,
  author =       {Brian Kernighan},
  title =        {Programming in C – A Tutorial},
  institution =  {Bell Laboratories},
  year =         {1974}
}

@Manual{Tantau2019,
  author =       {Till Tantau},
  title =        {The Ti\emph kZ and PGF Packages: Manual for version 3.1.3},
  institution =  {Institut für Theoretische Informatik, Universität zu Lübeck},
  year =         {2019},
  url =          {https://github.com/pgf-tikz/pgf}
}

@Book{Alley1996,
  author =       {Michael Alley},
  title =        {The Craft of Scientific Writing},
  publisher =    {Springer},
  year =         {1996},
  edition =      {Third Edition},
}

@Book{DowneyF13,
  author =       {R. G. Downey and M. R. Fellows},
  title =        {Fundamentals of Parameterized Complexity},
  series =       {Texts in Computer Science},
  publisher =    {Springer},
  year =         2013,
  doi =          {10.1007/978-1-4471-5559-1},
}

@Manual{biblatex,
  title =        {The \textsc{BibLaTeX} package},
  subtitle =     {Sophisticated Bibliographies in \LaTeX},
  author =       {Kime, Philip and Lehman, Philipp},
  url =          {https://github.com/plk/biblatex},
  urldate =      {2019-06-11},
  date =         {2018-10-30},
  version =      {3.12}
}

@Manual{varioref,
  title =        {The \textsc{varioref} package},
  subtitle =     {Intelligent page references},
  author =       {Mittelbach, Frank},
  url =          {http://www.ctan.org/pkg/varioref},
  urldate =      {2019-06-11},
  date =         {2016-02-16},
  version =      {1.5c}
}

@Manual{hyperref,
  title =        {The \textsc{hyperref} package},
  subtitle =     {Extensive support for hypertext in \LaTeX},
  author =       {Rahtz, Sebastian and Oberdiek, Heiko},
  url =          {https://github.com/ho-tex/hyperref},
  urldate =      {2019-06-11},
  date =         {2018-11-30},
  version =      {6.88e}
}

@Manual{babel,
  title =        {The \textsc{babel} package},
  subtitle =     {Multilingual support for Plain \TeX\ or \LaTeX},
  author =       {Braams, Johannes L. and Bezos López, Javier},
  url =          {http://www.ctan.org/pkg/babel},
  urldate =      {2019-06-11},
  date =         {2019-06-03},
  version =      {3.32}
}

@Manual{fontspec,
  title =        {The \textsc{fontspec} package},
  subtitle =     {Advanced font selection in Xe\LaTeX\ and Lua\LaTeX},
  author =       {Robertson, Will},
  url =          {http://www.ctan.org/pkg/fontspec},
  urldate =      {2019-06-11},
  version =      {2.7c}
}

@Manual{url,
  title =        {The \textsc{url} package},
  subtitle =     {Verbatim with \textsc{url}-sensitive line breaks},
  author =       {Arseneau, Donald},
  url =          {http://www.ctan.org/pkg/url},
  urldate =      {2019-06-11},
  date =         {2013-09-16},
  version =      {3.4}
}

@Manual{amsmath,
  title =        {The \textsc{amsmath} package},
  subtitle =     {\AmS\ mathematical facilities for \LaTeX},
  author =       {{The \LaTeX\ Team}},
  url =          {http://www.ams.org/tex/amslatex.html},
  urldate =      {2019-06-11}, 
  date =         {2017-09-02},
  version =      {2.17a}
}

@Book{Beutelspacher2009,
  title =        {„Das ist o.\,B.\,d.\,A.\ trivial!“: Tipps und Tricks zur
                  Formulierung mathematischer Gedanken (Mathematik für
                  Studienanfänger)},
  author =       {Albrecht Beutelspacher},
  year =         {2009},
  edition =      {Ninth, updated edition},
  publisher =    {Vieweg+Teubner Verlag},
  doi =          {10.1007/978-3-8348-9075-7},
}

\end{bibtex-entries}



% If you need to have an appendix (I advise against it), insert it
% here using, first, \appendix and then \chapter and then,
% possibly, \section. 
%
% \appendix
%
% \chapter{Technical Appendix}
%
% \section{Experimental Parameters} % possibly
%
% Again, I advise against using an appendix.


\end{document}

%  LocalWords:  LaTeX tex moretexcs Lübeck pdf uzl lualatex bibtex th
%  LocalWords:  TechReport Kernighan Lamport's Tantau's Tantau cls kZ
%  LocalWords:  Mustermann emacs oldschool pdflatex texmf utf biber
%  LocalWords:  biblatex Alphabetische Bibliographie Numerische VIIa
%  LocalWords:  varioref german Einleitung Beiträge dieser Arbeit xml
%  LocalWords:  Ergebnisse Verwandte Arbeiten Aufbau nucleotide VIIc
%  LocalWords:  ensembl amino phylogenetic Alexa Siri decrypt versa
%  LocalWords:  cryptographic pre nondeterministic deterministically
%  LocalWords:  Beutelspacher Untersuchungen zum genetischen sep llcc
%  LocalWords:  Beispiel tikz jpg png Alegrya Kasimir Malewitsch PGF
%  LocalWords:  Lamport Institut für Theoretische Informatik zu url
%  LocalWords:  Universität Springer DowneyF Downey Parameterized doi
%  LocalWords:  BibLaTeX Kime Philipp urldate Mittelbach hyperref Lua
%  LocalWords:  Rahtz Oberdiek Heiko Braams Bezos López fontspec Das
%  LocalWords:  Arseneau amsmath ist Tipps und zur Formulierung
%  LocalWords:  mathematischer Gedanken Mathematik Studienanfänger
%  LocalWords:  Albrecht Vieweg Teubner Verlag
%----------- Unbekanntes Dokument aus Abteilung 2
